% Options for packages loaded elsewhere
\PassOptionsToPackage{unicode}{hyperref}
\PassOptionsToPackage{hyphens}{url}
\documentclass[
  12pt,
]{article}
\usepackage{xcolor}
\usepackage[margin=1in]{geometry}
\usepackage{amsmath,amssymb}
\setcounter{secnumdepth}{5}
\usepackage{iftex}
\ifPDFTeX
  \usepackage[T1]{fontenc}
  \usepackage[utf8]{inputenc}
  \usepackage{textcomp} % provide euro and other symbols
\else % if luatex or xetex
  \usepackage{unicode-math} % this also loads fontspec
  \defaultfontfeatures{Scale=MatchLowercase}
  \defaultfontfeatures[\rmfamily]{Ligatures=TeX,Scale=1}
\fi
\usepackage{lmodern}
\ifPDFTeX\else
  % xetex/luatex font selection
\fi
% Use upquote if available, for straight quotes in verbatim environments
\IfFileExists{upquote.sty}{\usepackage{upquote}}{}
\IfFileExists{microtype.sty}{% use microtype if available
  \usepackage[]{microtype}
  \UseMicrotypeSet[protrusion]{basicmath} % disable protrusion for tt fonts
}{}
\makeatletter
\@ifundefined{KOMAClassName}{% if non-KOMA class
  \IfFileExists{parskip.sty}{%
    \usepackage{parskip}
  }{% else
    \setlength{\parindent}{0pt}
    \setlength{\parskip}{6pt plus 2pt minus 1pt}}
}{% if KOMA class
  \KOMAoptions{parskip=half}}
\makeatother
\usepackage{graphicx}
\makeatletter
\newsavebox\pandoc@box
\newcommand*\pandocbounded[1]{% scales image to fit in text height/width
  \sbox\pandoc@box{#1}%
  \Gscale@div\@tempa{\textheight}{\dimexpr\ht\pandoc@box+\dp\pandoc@box\relax}%
  \Gscale@div\@tempb{\linewidth}{\wd\pandoc@box}%
  \ifdim\@tempb\p@<\@tempa\p@\let\@tempa\@tempb\fi% select the smaller of both
  \ifdim\@tempa\p@<\p@\scalebox{\@tempa}{\usebox\pandoc@box}%
  \else\usebox{\pandoc@box}%
  \fi%
}
% Set default figure placement to htbp
\def\fps@figure{htbp}
\makeatother
% definitions for citeproc citations
\NewDocumentCommand\citeproctext{}{}
\NewDocumentCommand\citeproc{mm}{%
  \begingroup\def\citeproctext{#2}\cite{#1}\endgroup}
\makeatletter
 % allow citations to break across lines
 \let\@cite@ofmt\@firstofone
 % avoid brackets around text for \cite:
 \def\@biblabel#1{}
 \def\@cite#1#2{{#1\if@tempswa , #2\fi}}
\makeatother
\newlength{\cslhangindent}
\setlength{\cslhangindent}{1.5em}
\newlength{\csllabelwidth}
\setlength{\csllabelwidth}{3em}
\newenvironment{CSLReferences}[2] % #1 hanging-indent, #2 entry-spacing
 {\begin{list}{}{%
  \setlength{\itemindent}{0pt}
  \setlength{\leftmargin}{0pt}
  \setlength{\parsep}{0pt}
  % turn on hanging indent if param 1 is 1
  \ifodd #1
   \setlength{\leftmargin}{\cslhangindent}
   \setlength{\itemindent}{-1\cslhangindent}
  \fi
  % set entry spacing
  \setlength{\itemsep}{#2\baselineskip}}}
 {\end{list}}
\usepackage{calc}
\newcommand{\CSLBlock}[1]{\hfill\break\parbox[t]{\linewidth}{\strut\ignorespaces#1\strut}}
\newcommand{\CSLLeftMargin}[1]{\parbox[t]{\csllabelwidth}{\strut#1\strut}}
\newcommand{\CSLRightInline}[1]{\parbox[t]{\linewidth - \csllabelwidth}{\strut#1\strut}}
\newcommand{\CSLIndent}[1]{\hspace{\cslhangindent}#1}
\setlength{\emergencystretch}{3em} % prevent overfull lines
\providecommand{\tightlist}{%
  \setlength{\itemsep}{0pt}\setlength{\parskip}{0pt}}
\usepackage{url}
\usepackage{setspace}
%\singlespacing
%\onehalfspacing
\doublespacing
\usepackage{lineno}
\linenumbers
\usepackage[belowskip=0pt,aboveskip=0pt]{caption}
\usepackage{relsize}
\usepackage{float}
\usepackage{lscape}
\usepackage{longtable}
\usepackage{amsmath,rotating}
\usepackage[scanall]{psfrag}
\usepackage{bm}
\usepackage{caption,graphics}
\usepackage{graphicx}
\usepackage{sectsty}
\usepackage{color}
\usepackage{fancyhdr}
\usepackage{xspace}
\usepackage{textcomp}
\usepackage{upgreek}
\renewcommand\figurename{Fig.}
\captionsetup{labelsep=period, singlelinecheck=false}
\newcommand{\changesize}[1]{\fontsize{#1pt}{#1pt}\selectfont}
\renewcommand{\arraystretch}{1.5}
%\renewcommand\theadfont{}

\newcommand{\Fmsy}{\ensuremath{F_{\text{MSY}}}\xspace}
\newcommand{\Fspr}[1]{\ensuremath{F_{\text{{#1}\%}}}\xspace}
\newcommand{\afrb}{Alaska Fishery Research Bulletin\xspace}
\newcommand{\ajms}{African Journal of Marine Science\xspace}
\newcommand{\amb}{Advances in Marine Biology\xspace}
\newcommand{\bms}{Bulletin of Marine Science\xspace}
\newcommand{\bjssf}{Bulletin of the Japanese Society of Scientific Fisheries\xspace}
\newcommand{\cb}{Conservation Biology\xspace}
\newcommand{\cjfas}{Canadian Journal of Fisheries and Aquatic Sciences\xspace}
\newcommand{\ea}{Ecological Applications\xspace}
\newcommand{\eer}{Evolutionary Ecology Research\xspace}
\newcommand{\elet}{Ecology Letters\xspace}
\newcommand{\emod}{Ecological Modelling\xspace}
\newcommand{\ebf}{Environmental Biology of Fishes\xspace}
\newcommand{\ff}{Fish and Fisheries\xspace}
\newcommand{\fo}{Fisheries Oceanography\xspace}
\newcommand{\fr}{Fisheries Research\xspace}
\newcommand{\fb}{Fishery Bulletin\xspace}
\newcommand{\ijms}{ICES Journal of Marine Science\xspace}
\newcommand{\iccat}{Collective Volume of Scientific Papers ICCAT\xspace}
\newcommand{\jae}{Journal of Animal Ecology\xspace}
\newcommand{\jai}{Journal of Applied Ichthyology\xspace}
\newcommand{\jdc}{Journal Du Conseil International Pour L'exploration De La Mer\xspace}
\newcommand{\jdcp}{Journal Du Conseil Permanent International Pour L'exploration De La Mer\xspace}
\newcommand{\jembe}{Journal of Experimental Marine Biology and Ecology\xspace}
\newcommand{\jfb}{Journal of Fish Biology\xspace}
\newcommand{\jsr}{Journal of Sea Research\xspace}
\newcommand{\jtb}{Journal of Theoretical Biology\xspace}
\newcommand{\jfrbc}{Journal of the Fisheries Research Board of Canada\xspace}
\newcommand{\jnwafs}{Journal of Northwest Atlantic Fisheries Science\xspace}
\newcommand{\mcf}{Marine and Coastal Fisheries: Dynamics, Management, and Ecosystem Science\xspace}
\newcommand{\mb}{Marine Biology\xspace}
\newcommand{\meps}{Marine Ecology Progress Series\xspace}
\newcommand{\mfr}{Marine Fisheries Review\xspace}
\newcommand{\mpb}{Marine Pollution Bulletin\xspace}
\newcommand{\najfm}{North American Journal of Fisheries Management\xspace}
\newcommand{\nzjmfr}{New Zealand Journal of Marine and Freshwater Research\xspace}
\newcommand{\pnas}{Proceedings of the National Academy of Sciences USA\xspace}
\newcommand{\rpvrciemm}{Rapports et Proc\`es-Verbaux des R\'eunions. Conseil Internationale pour l'Exploration de la Mer\xspace}
\newcommand{\rpvrcpiemm}{Rapports et Proc\`es-Verbaux des R\'eunions. Conseil Permanent Internationale pour l'Exploration de la Mer\xspace}
\newcommand{\rfbf}{Reviews in Fish Biology and Fisheries\xspace}
\newcommand{\sajms}{South African Journal of Marine Science\xspace}
\newcommand{\tafs}{Transactions of the American Fisheries Society\xspace}

\newcommand{\anzjs}{Australian \& New Zealand Journal of Statistics\xspace}
\newcommand{\as}{Applied Statistics\xspace}
\newcommand{\csda}{Computational Statistics \& Data Analysis\xspace}
\newcommand{\ees}{Environmental and Ecological Statistics\xspace}
\newcommand{\jas}{Journal of Applied Statistics\xspace}
\newcommand{\jabes}{Journal of Agricultural, Biological, and Environmental Statistics\xspace}
\newcommand{\jasa}{Journal of the American Statistical Association\xspace}
\newcommand{\jrssb}{Journal of the Royal Statistical Society. Series B\xspace}
\newcommand{\sm}{Statistics in Medicine}

\usepackage{booktabs}
\usepackage{longtable}
\usepackage{array}
\usepackage{multirow}
\usepackage{wrapfig}
\usepackage{float}
\usepackage{colortbl}
\usepackage{pdflscape}
\usepackage{tabu}
\usepackage{threeparttable}
\usepackage{threeparttablex}
\usepackage[normalem]{ulem}
\usepackage{makecell}
\usepackage{xcolor}
\usepackage{bookmark}
\IfFileExists{xurl.sty}{\usepackage{xurl}}{} % add URL line breaks if available
\urlstyle{same}
\hypersetup{
  pdftitle={An investigation of factors affecting inferences from and reliability of state-space age-structured assessment models},
  pdfauthor={Timothy J. Miller1,2; Gregory L. Britten3; Elizabeth N. Brooks2; Gavin Fay4; Alexander C. Hansell2; Christopher M. Legault2; Chengxue Li2; Brandon Muffley5; Brian C. Stock6; John Wiedenmann7},
  hidelinks,
  pdfcreator={LaTeX via pandoc}}

\title{An investigation of factors affecting inferences from and
reliability of state-space age-structured assessment models}
\author{Timothy J. Miller\textsuperscript{1,2} \and Gregory L.
Britten\textsuperscript{3} \and Elizabeth N.
Brooks\textsuperscript{2} \and Gavin
Fay\textsuperscript{4} \and Alexander C.
Hansell\textsuperscript{2} \and Christopher M.
Legault\textsuperscript{2} \and Chengxue
Li\textsuperscript{2} \and Brandon Muffley\textsuperscript{5} \and Brian
C. Stock\textsuperscript{6} \and John Wiedenmann\textsuperscript{7}}
\date{19 December, 2025}

\begin{document}
\maketitle

\(^1\)corresponding author:
\href{mailto:timothy.j.miller@noaa.gov}{\nolinkurl{timothy.j.miller@noaa.gov}}\\
\(^2\)Northeast Fisheries Science Center, Woods Hole Laboratory, 166
Water Street, Woods Hole, MA 02543 USA\\
\(^3\)Biology Department, Woods Hole Oceanographic Institution, 266
Woods Hole Rd. Woods Hole, MA, USA\\
\(^4\)Department of Fisheries Oceanography, School for Marine Science
and Technology, University of Massachusetts Dartmouth, 836 S Rodney
French Boulevard, New Bedford, MA 02740, USA\\
\(^5\)Mid-Atlantic Fishery Management Council, 800 North State Street,
Suite 201, Dover, DE 19901 USA\\
\(^6\)Institute of Marine Research, Nye Flødevigveien 20, 4817 His,
Norway\\
\(^7\)Department of Ecology, Evolution, and Natural Resources. Rutgers
University\\

\pagebreak

\textbf{keywords:} stock assessment, state-space, model selection, bias,
convergence, retrospective patterns

\subsection*{Abstract}\label{abstract}
\addcontentsline{toc}{subsection}{Abstract}

State-space models have been promoted as the next-generation of
fisheries stock assessment and evaluation of their reliability is
needed. We simulated operating models that varied fishing pressure,
magnitude of observation error, and sources of process error. For each
operating model, we fit a range of estimating models with correct and
incorrect configurations. We measured reliability of estimating models
by convergence rate, accuracy of AIC-based model selection, estimation
bias, and magnitude of retrospective patterns. All reliability measures
were generally better with lower observation error, contrast in fishing
pressure over time, and when median natural mortality rate is known. The
magnitude of the log-likelihood gradients was not a reliable indicator
of convergence. AIC can generally distinguish process error source with
lower observation error and higher true process error variability.
Distinguishing the stock recruit relationship with AIC required large
contrast in spawning biomass and low recruitment variation, but bias in
stock-recruit parameter estimation was prevalent. Retrospective patterns
were not large for mis-specified models. These findings improve our
understanding of when results from state space models will be reliable.

\pagebreak

\section*{Introduction}\label{introduction}
\addcontentsline{toc}{section}{Introduction}

Application of state-space models in fisheries stock assessment and
management has expanded dramatically within the International Council
for the Exploration of the Sea (ICES), Canada, and the Northeast US
(Nielsen and Berg 2014; Cadigan 2016; Pedersen and Berg 2017; Stock and
Miller 2021). State-space models treat latent population characteristics
as statistical time series with periodic observations that also may have
error due to sampling or other measurement properties. Traditional
assessment models may use state-space approaches to account for temporal
variability in population characteristics (Legault and Restrepo 1999;
Methot and Wetzel 2013), but these models treat the annual parameters as
penalized fixed effect parameters where the variance parameters
controlling the penalties are assumed known (Thorson and Minto 2015).
Modern state-space models can estimate the annually varying parameters
as random effects with variance parameters estimated using maximum
marginal likelihood or corresponding Bayesian approaches. These
random-effects approaches are considered best practice and are
recommended for the next generation of stock assessment models (Hoyle et
al. 2022; Punt 2023).

State-space stock assessment models, with nonlinear functions of latent
parameters and multiple types of observations with varying
distributional assumptions, are one of the most complex examples of this
analytical approach. Statistical aspects of state-space models and their
application within fisheries have been studied extensively, but previous
work has focused primarily on linear and Gaussian state-space models
(Aeberhard et al. 2018; {Auger-Méthé et al.} 2021). Therefore, current
understanding of the reliability of state-space models does not extend
to usage for stock assessment.

As state-space models provide greater flexibility by allowing multiple
processes to vary as random effects (Nielsen and Berg 2014; Aeberhard et
al. 2018; Stock et al. 2021), one of the most immediate questions
regards the implications of mis-specification among alternative sources
of process error. Incorrect treatment of population attributes as
temporally varying (Trijoulet et al. 2020; Liljestrand et al. 2024)
could lead to misidentification of stock status and biased population
estimates, ultimately impacting fisheries management decisions (Legault
and Palmer 2016; Szuwalski et al. 2018; Cronin-Fine and Punt 2021).
Furthermore, biological, fishery, and observational processes are often
confounded in catch-at-age data, which may adversely affect the ability
to distinguish between true process variability and observational error
(Punt et al. 2014; Stewart and Monnahan 2017; Cronin-Fine and Punt 2021;
Fisch et al. 2023; Li et al. 2025a).

Li et al. (2024) conducted a full-factorial simulation-estimation study
to assess model reliability when confounding random-effects processes
(numbers-at-age, fishery selectivity, and natural mortality) were
included. Their results suggest that while state-space models can
generally identify sources of process error, overly complex models, even
when misspecified (i.e., incorporating process error that did not exist
in reality), often performed similarly to correctly specified models,
with little to no bias in key management quantities. Similarly,
Liljestrand et al. (2024) found little downside in assuming process
error in recruitment or selectivity, even when it was absent.

Despite increasing research on state space assessment models, several
uncertainties in state space assessment modeling remain. First,
confounding processes that can be treated as random effects in the model
have not been thoroughly examined or tested within a
simulation-estimation framework. Second, previous studies relied on
operating models conditioned on specific fisheries, limiting their
generalizability (Liljestrand et al. 2024; Li et al. 2025a). In
particular, the effects of observation error and underlying fishing
history have not been fully isolated in simulation study designs, making
it challenging to disentangle the interplay between process and
observation error magnitudes, as demonstrated in Fisch et al. (2023).
Third, explicitly modeling stock-recruit relationships (SRRs) as
mechanistic drivers of population dynamics is promising (Fleischman et
al. 2013; Pontavice et al. 2022), but reliability of inferences within
integrated state-space age-structured models has not been evaluated.
Evidence from other studies suggests that when both process and
observation errors are unknown, estimating density dependence parameters
becomes highly uncertain (Knape 2008; Polansky et al. 2009). In
particular, Knape (2008) demonstrated that stronger density dependence
becomes increasingly difficult to estimate in the presence of
observation error. Therefore, it is crucial to assess whether
density-dependent mechanisms can be estimated with sufficient precision
for use in fisheries management (Auger-Méthé et al. 2016). Finally,
although the importance of autocorrelation in process errors is
recognized, investigations of the ability to distinguish state-space
assessment models with and without autocorrelation and whether such
misspecification is detrimental to estimation of important population
metrics are lacking (Johnson et al. 2016; Xu et al. 2019).

In the present study, we conduct a simulation study with operating
models (OMs) varying by degree of observation error, source and
variability of process error, and fishing history. The simulations from
these OMs are fitted with estimation models (EMs) that make alternative
assumptions for sources of process error, whether an SRR was estimated,
and whether natural mortality is estimated. Given the confounding nature
of process errors, developing diagnostic tools to detect model
misspecification is of great scientific interest and could aid the next
generation of stock assessments ({Auger-Méthé et al.} 2021). We evaluate
whether OM and EM attributes affect rates of convergence and the ability
of Akaike Information Criterion (AIC) to correctly determine the source
of process error or the existence of an SRR. We also evaluate effects of
OM and EM attributes on magnitude of retrospective patterns and bias in
estimation of parameters and other model outputs important for
management.

\section*{Methods}\label{methods}
\addcontentsline{toc}{section}{Methods}

We used the Woods Hole Assessment Model (WHAM) to configure OMs and EMs
in our simulation study (Stock and Miller 2021; Miller et al. 2025).
WHAM is an R package freely available via a Github repository and is
built on the Template Model Builder package (Kristensen et al. 2016).
For this study we used
\href{https://github.com/timjmiller/wham/tree/77bbd946e4881216a439933473d1c58b21c270c3}{version
1.0.6.9000, commit 77bbd94}. WHAM has also been used to configure OMs
and EMs for closed loop simulations evaluating index-based assessment
methods (Legault et al. 2023) and is currently used or accepted for use
in management of numerous Northeast United States (NEUS) fish stocks
(e.g., NEFSC 2022a, 2022b; NEFSC 2024).

We completed a simulation study with a number of OMs that can be
categorized based on where process error random effects were assumed. R
OMs assume process error for recruitment only. Other OM categories
assume recruitment process errors along with process errors for apparent
survival (R+S), natural mortality (R+M), fleet selectivity (R+Sel), or
index catchability (R+q). We refer to the R+S OMs as modeling apparent
survival because on log-scale the random effects are additive to the
total mortality (fishing and natural mortality) between numbers at age,
thus they modify the survival term. For each OM, assumptions about the
magnitude of the variance of process errors and observations are
required and the values we used were based on a review of the range of
estimates from NEUS assessments using WHAM.

In total, we configured 72 OMs with alternative assumptions about the
source and magnitude of process errors, magnitude of observation error
in indices and age composition data, and contrast in fishing pressure
over time. For each OM, we simulated 100 time series of abundance at age
with process errors, and for each realized time series, we simulated
observation data sets. For each data set, we fitted a number of EMs that
differed in assumptions about the source of process errors, whether
natural mortality (or the median for models with process error in
natural mortality) was estimated, and whether a Beverton-Holt SRR was
estimated within the EM. Details of each of the OMs and EMs are
described below. We did not use the log-normal bias-correction feature
for process errors or observations described by Stock and Miller (2021)
for OMs and EMs to simplify interpretation of the study results (Li et
al. 2025b). All code we used to perform the simulation study and
summarize results can be found at
\url{https://github.com/timjmiller/SSRTWG/tree/main/Project_0/code}.

\subsection*{Operating models}\label{operating-models}
\addcontentsline{toc}{subsection}{Operating models}

\subsubsection*{Population}\label{population}
\addcontentsline{toc}{subsubsection}{Population}

We intended the population demographics and observation types to
represent a general NEUS groundfish stock. The population consists of 10
age classes, ages 1 to 10+, with the last being a plus group that
accumulates ages 10 and older. The maturity at age was a logistic curve
with \(a_{50}\) = 2.89 and slope = 0.88 (Figure \ref{om_inputs_fig}, top
left). Weight at age (\(W_a\)) was generated with a von Bertalanffy
growth function defining length at age: \[
L_a = L_{\infty}\left(1 - e^{-k(a - t_0)}\right),
\] where \(t_0 = 0\), \(L_\infty = 85\), and \(k = 0.3\), and a
length-weight relationship such that \[
W_a = \theta_1 L_a^{\theta_2},
\] where \(\theta_1 = e^{-12.1}\) and \(\theta_2 = 3.2\) (Figure
\ref{om_inputs_fig}, top right).

We assumed a Beverton-Holt SRR with constant pre-recruit mortality
parameters for all OMs. We assume spawning occurs annually 0.25 of each
year and recruitment at age 1 (\(N_{1,y}\)). All biological inputs to
calculations of spawning stock biomass (SSB) per recruit (i.e., weight,
maturity, and natural mortality at age) are constant in the R+S, R+Sel,
and R+q process error OMs. Therefore, steepness and unfished recruitment
are also constant over the time period for those OMs (Miller and Brooks
2021). We assumed a value of 0.2 for the natural mortality rate in OMs
without process errors on natural mortality. We specified unfished
recruitment equal to \(e^{10}\) and \(\Fmsy = F_{40\%} = 0.348\), which
equates to a steepness of 0.69 and \(a=0.60\) and
\(b = 2.4 \times 10^{-5}\) for the Beverton-Holt parameterization \[
N_{1,y} = \frac{a \text{SSB}_{y-1}}{1 + b \text{SSB}_{y-1}} 
\] (Figure \ref{om_inputs_fig}, bottom right). For OMs with time-varying
random effects for natural mortality, steepness is not constant.
However, we used the same \(a\) and \(b\) parameters as other OMs, which
equates to a steepness and R0 at the median of the time series process
for natural mortality. Similarly, for OMs with time-varying random
effects for fishery selectivity, \Fmsy also varies temporally, so
equilibrium conditions for these OMs are defined for mean selectivity
parameters.

We used two fishing scenarios for OMs. In the first scenario, the stock
experiences overfishing at 2.5\Fmsy for the first 20 years followed by
fishing at \Fmsy for the last 20 years (denoted
\(2.5\Fmsy \rightarrow \Fmsy\)). In the second scenario, the stock is
fished at \Fmsy for the entire time period (40 years). The magnitude of
the overfishing assumptions is based on average estimates of overfishing
for NEUS groundfish stocks from Wiedenmann et al. (2019) and similar to
the approach in Legault et al. (2023). The second scenario represents
the ideal situation where the stock is fished at an optimal level, but
provides less contrast in stock sizes over time. We specified initial
population abundance at age at the equilibrium distribution that
corresponds to fishing at either \(F = 2.5\Fmsy\) or \(F = \Fmsy\). This
implies that, for a deterministic model, the abundance at age would not
change from year to year at the beginning of the time series.

\subsubsection*{Fleets}\label{fleets}
\addcontentsline{toc}{subsubsection}{Fleets}

We assumed a single fleet operating year round for catch observations
with logistic selectivity (\(a_{50} = 5\) and slope = 1; Figure
\ref{om_inputs_fig}, bottom left). This selectivity was used to define
\Fmsy for the Beverton-Holt SRR parameters above. We assumed a
logistic-normal distribution with no correlation on the multivariate
normal scale for the corresponding annual age-composition observations.

\subsubsection*{Indices}\label{indices}
\addcontentsline{toc}{subsubsection}{Indices}

Two time series of fishery-independent surveys measured in numbers are
generated for the entire 40 year period with one occurring in the spring
(0.25 of each year) and one in the fall (0.75 of each year),
representing current bottom trawl surveys conducted in the NEUS.
Catchability of both surveys are assumed to be 0.1. Like the fishing
fleet, we assumed logistic selectivity for both indices (\(a_{50} = 5\)
and slope = 1) and a logistic-normal distribution with no correlation on
the multivariate normal scale for the annual age-composition
observations.

\subsubsection*{Observation Uncertainty}\label{observation-uncertainty}
\addcontentsline{toc}{subsubsection}{Observation Uncertainty}

The standard deviation for log-aggregate catch was 0.1 for all OMs, a
common assumption for commercial removals in NEUS stock assessments. Two
levels of observation error variance (high and low) were specified for
indices and all age composition observations (both indices and catch).
The low uncertainty specification assumed a standard deviation of 0.1
for both series of log-aggregate index observations, and the standard
deviation of the logistic-normal for age composition observations was
0.3. In the high uncertainty specification, the standard deviation for
log-aggregate indices was 0.4 and that for the age composition
observations was 1.5. The low standard deviation for index observations
is typical for fish stocks that are consistently sampled across survey
stations whereas the high value is typical for more sporadically sampled
stocks. The standard deviations for the age composition observations
were determined from the range of values estimated from WHAM fits to
NEUS stocks that assumed the logistic-normal model. For all EMs, the
standard deviation for log-aggregate observations was assumed known
whereas that for the logistic-normal age composition observations was
estimated.

\subsubsection*{Operating models with random effects on numbers at
age}\label{operating-models-with-random-effects-on-numbers-at-age}
\addcontentsline{toc}{subsubsection}{Operating models with random
effects on numbers at age}

For operating models with random effects on recruitment only and also on
apparent survival (R, R+S), we assumed marginal standard deviations for
recruitment of \(\sigma_R \in \{0.5,1.5\}\). The marginal standard
deviations for apparent survival random effects at older age classes
were \(\sigma_{2+} \in \{0,0.25, 0.5\}\). The full factorial combination
of these process error assumptions (\(2\times 3\) levels) and scenarios
for fishing history (2 levels) and observation error (2 levels)
scenarios described above results in 24 different R
(\(\sigma_{2+} = 0\)) and R+S operating models (Table
\ref{naa_om_table}).

\subsubsection*{Operating models with random effects on natural
mortality}\label{operating-models-with-random-effects-on-natural-mortality}
\addcontentsline{toc}{subsubsection}{Operating models with random
effects on natural mortality}

All R+M OMs treat natural mortality as constant across age, but with
annually varying random effects. WHAM treats natural mortality as a
log-transformed parameter \[
\log M_{y,a} = \mu_{M} + \epsilon_{M,y}
\] that is a linear combination of a mean log-natural mortality
parameter that is constant across ages (\(\mu_{M} = \log(0.2)\)) and any
annual random effects are marginally distributed as
\(\epsilon_{M,y} \sim \text{N}\left(0,\sigma_M^2\right)\). The marginal
standard deviations we assumed for log natural mortality random effects
were \(\sigma_M \in \{0.1, 0.5\}\) and the random effects were either
uncorrelated or first-order autoregressive (AR1,
\(\rho_M \in \{0,0.9\}\)). Uncorrelated random effects were also
included on recruitment with \(\sigma_R = 0.5\) (hence, we denote these
OMs as R+M). The full factorial combination of these process error
assumptions and fishing history (2 levels) and observation error (2
levels) scenarios described above results in 16 different R+M OMs (Table
\ref{M_om_table}).

\subsubsection*{Operating models with random effects on fleet
selectivity}\label{operating-models-with-random-effects-on-fleet-selectivity}
\addcontentsline{toc}{subsubsection}{Operating models with random
effects on fleet selectivity}

WHAM treats each selectivity parameter \(s\) as a logit-transformed
parameter \[
\log\left(\frac{p_{s,y}-l_{s}}{u_{s}-p_{s,y}}\right) = \mu_s + \epsilon_{s,y}
\] that is a linear combination of a mean \(\mu_s\) and any annual
random effects marginally distributed as
\(\epsilon_{s,y} \sim \text{N}\left(0,\sigma_s^2\right)\), where the
lower and upper bounds of the parameter (\(l_s\) and \(u_s\)) can be
specified by the user. All selectivity parameters (\(a_{50}\) and slope
parameters) were bounded by \(s_l = 0\) and \(s_u = 10\) for all OMs and
EMs. The marginal standard deviations we assumed for logit scale random
effects were \(\sigma_s \in \{0.1, 0.5\}\) and AR1 autocorrelation
parameters of \(\rho_s \in \{0,0.9\}\). Like R+M OMs, the full factorial
combination of these process error assumptions (2x2 levels) and
scenarios described above for fishing history (2 levels) and observation
error (2 levels) results in 16 different R+Sel OMs (Table
\ref{sel_om_table}).

\subsubsection*{Operating models with random effects on index
catchability}\label{operating-models-with-random-effects-on-index-catchability}
\addcontentsline{toc}{subsubsection}{Operating models with random
effects on index catchability}

Like selectivity parameters, WHAM treats catchability for an index \(i\)
as a logit-transformed parameter \[
\log\left(\frac{q_{i,y}-l_{i}}{u_{i}-q_{i,y}}\right) = \mu_i + \epsilon_{i,y}
\] that is a linear combination of a mean \(\mu_i\) and any annual
random effects marginally distributed as
\(\epsilon_{i,y} \sim \text{N}\left(0,\sigma_i^2\right)\) where the
lower and upper bounds of the catchability (\(l_i\) and \(u_i\)) can be
specified by the user. We assumed bounds of 0 and 1000 for all OMs and
EMs. For all OMs and EMs with process errors on catchability, the
temporal variation only applies to the first index, which could be
interpreted as capturing some unmeasured seasonal process that affects
availability to the survey. The marginal standard deviations we assumed
for logit scale random effects were \(\sigma_i \in \{0.1, 0.5\}\) and
AR1 autocorrelation parameters of \(\rho_i \in \{0,0.9\}\). Like R+M and
R+Sel OMs, the full factorial combination of these process error
assumptions and fishing history (2 levels) and observation error (2
levels) scenarios described above results in 16 different R+q OMs (Table
\ref{q_om_table}).

\subsection*{Estimation models}\label{estimation-models}
\addcontentsline{toc}{subsection}{Estimation models}

For each of the data sets simulated from an OM, 20 EMs were fit. A total
of 32 different EMs were fit across OMs where the subset of 20 depended
on the source of process error in the OM (Table \ref{em_table}). The EMs
have different assumptions about the source of process error (R+S, R+M,
R+Sel, R+q) and whether or not 1) there is temporal autocorrelation, 2)
a Beverton-Holt SRR is estimated, and 3) the natural mortality rate
(\(\mu_M\), the constant or mean on log scale for R+M EMs) is estimated.
For simplicity we refer to the derived estimate \(e^{\mu_M}\) as the
median natural mortality rate regardless of whether natural mortality
random effects are estimated in the EM.

Subsets of 20 EMs in Table \ref{em_table} were fit to simulated data
sets from each of the OM process error sources. For R and R+S OMs,
fitted EMs had matching process error assumptions as well as R+Sel, R+M,
and R+q assumptions without autocorrelation. For other OM process error
sources, we fit EMs with correct process error assumptions, the correct
process error source but incorrect correlation assumption, and the
incorrect process error source without autocorrelation. As such, EMs
were configured correctly for the OM, or they had mis-specification in
assumptions of process error autocorrelation, the source of process
error, and(or) the SRR (Beverton-Holt or none).

The maturity at age, weight at age for catch and SSB, and observation
error standard deviations for aggregate catch and indices were all
assumed known at the true values. However, the variance parameters for
the logistic-normal distributions for age composition observations were
estimated in the EMs.

\subsection*{Measures of reliability}\label{measures-of-reliability}
\addcontentsline{toc}{subsection}{Measures of reliability}

\subsubsection*{Convergence}\label{convergence}
\addcontentsline{toc}{subsubsection}{Convergence}

The first measure of reliability we investigated was frequency of
convergence when fitting each EM to the simulated data sets. There are
various ways to assess convergence of the fit (e.g., Carvalho et al.
2021; Kapur et al. 2025), but given the importance of estimates of
uncertainty when using assessment models in management, we estimated
probability of convergence as measured by occurrence of a
positive-definite Hessian matrix at the optimized negative
log-likelihood that could be inverted (i.e., providing Hessian-based
standard error estimates). We also provide results in the Supplementary
Materials for convergence defined by the maximum absolute gradient
\textless{} \(1^{-6}\) and the maximum of the absolute gradient values
for all fits of a given EM to all simulated data sets from a given OM
that produced Hessian-based standard errors for all estimated fixed
effects. This provides an indication of how poor the calculated
gradients can be, but still presumably converged adequately enough for
parameter inferences.

\subsubsection*{AIC for model selection}\label{aic-for-model-selection}
\addcontentsline{toc}{subsubsection}{AIC for model selection}

We investigated the reliability of AIC-based model selection for two
purposes. First, we analyzed selection of each process error model
source (R, R+S, R+M, R+Sel, R+q) using marginal AIC. For a given OM
simulated data set, we compared AIC for EMs with different process error
assumption conditional on whether median natural mortality rate and the
Beverton-Holt SRR were estimated. Second, we analyzed AIC-based
selection between EMs with and without the Beverton-Holt SRR assumed.
Contrast in fishing pressure and time series with recruitment at low
stock size have been shown to improve estimation of SRR parameters
(Magnusson and Hilborn 2007; Conn et al. 2010). Our preliminary
inspections indicated generally poor performance of AIC in determining
the Beverton-Holt SRR model for a given set of OM factors (including
contrast in fishing pressure), even when the EM was configured with the
correct process error source. Therefore, we conditioned on the EMs
having the correct process error assumption and also considered the
effect of the log-standard deviation of the true log(SSB)
(\(\log \text{SD}_{\text{SSB}}\); similar to the log of the coefficient
of variation for SSB) on model selection since simulations with realized
SSB producing low and high recruitment would have larger variation in
realized SSB.

All model selection results condition only on completion of the
optimization process without failure for all of the compared EMs. We did
not condition on convergence as defined above because optimization could
correctly determine an inappropriate process error assumption by
estimating variance parameters at the lower bound of zero. Such an
optimization could indicate poor convergence but the likelihood would be
equivalent to that without the mis-specified random effects and the AIC
would be appropriately higher because more (variance) parameters were
estimated. All other measures of reliability described below (bias and
Mohn's \(\rho\)) use these same criteria for inclusion of EM fits in the
summarized results.

\subsubsection*{Bias}\label{bias}
\addcontentsline{toc}{subsubsection}{Bias}

We also investigated bias in estimation of various model attributes as a
measure of reliability. For a given model attribute we calculated the
relative error \begin{equation}\label{relerror}
\text{RE}\left(\theta_j\right) = \frac{\widehat \theta_j - \theta_j}{\theta_j}
\end{equation} from fitting a given EM to simulated data set \(j\)
configured for a given OM where \(\widehat \theta_j\) and \(\theta_j\)
are the estimated and true values for simulation \(j\). We analyzed
simulation results for estimates of terminal year SSB and recruitment,
Beverton-Holt SRR parameters (\(a\) and \(b\)), and median natural
mortality rate.

\subsubsection*{\texorpdfstring{Mohn's
\(\rho\)}{Mohn's \textbackslash rho}}\label{mohns-rho}
\addcontentsline{toc}{subsubsection}{Mohn's \(\rho\)}

Finally, we investigated presence of retrospective patterns in fitted
models as a measure of reliability. We calculated Mohn's \(\rho\) for
SSB, fishing mortality (averaged over all age classes), and recruitment
for each EM fit to each OM simulated data set (Mohn 1999). We fit
\(P = 7\) peels to each simulated data set and calculated Mohn's
\(\rho\) for a given attribute \(\theta\) as
\begin{equation}\label{mohns_rho}
\rho\left(\theta\right) = \frac{1}{P}\sum^P_{p=1} \frac{\widehat \theta_{Y-p,Y-p} - \widehat\theta_{Y-p,Y}}{\widehat\theta_{Y-p,Y}}
\end{equation} where \(Y\) is last year of the full set of observations
and \(\widehat \theta_{y,y'}\) is the estimate for attribute \(\theta\)
in year \(y\) from a model fit using data up to year \(y'\geq y\). Thus,
\(\theta\) terms where \(y'=Y\) refer to estimates from the fit to all
years of data.

\subsubsection*{Summarizing results across OM and EM
attributes}\label{summarizing-results-across-om-and-em-attributes}
\addcontentsline{toc}{subsubsection}{Summarizing results across OM and
EM attributes}

Because the OM and EM attributes that we investigated are numerous, we
used two methods to summarize the most important factors for differences
in results within a given OM process error source. The first method was
fitting regression models with the response being each of the measures
of reliability described above and predictor variables were defined
based on OM and EM characteristics (e.g., MacKinnon et al. 1995; Wang et
al. 2017; Harwell et al. 2018). For the binary indicators of convergence
and AIC-based selection of an SRR, we performed logistic regressions.
For indicators of AIC-based selection of EM process error source
(multiple categories) we performed multinomial regressions. For other
measures of reliability we fit linear regression models to transformed
responses. Because relative errors (Eq. \ref{relerror}) and Mohn's
\(\rho\) for the various parameters are bounded below at -1, we used a
transformation of these values
\begin{equation}\label{bias_regression_response}
y_j = \log\left[f\left(\widehat \theta_j,\theta_j\right)+1\right]
\end{equation} where \(f\) is either the relative error (Eq.
\ref{relerror}) or Mohn's \(\rho\) (Eq. \ref{mohns_rho}) for simulation
\(j\), so that values are unbounded. For relative errors, \(y_j\) is the
log-scale error. We omitted simulations where estimated attributes equal
to zero (RE = -1). For all regressions we fit separate models with just
individual OM and EM factors included, with all factors included, with
all second order interactions, and with all third order interactions.
For the multinomial regression, we used the \verb|vglm| function from
the VGAM package (Yee 2008; Yee 2015). We tabulated percent reduction in
residual deviance for each of the regression fits. We did not perform
formal statistical analyses of effects of OM and EM attributes on
results (e.g., ANOVA) because of the lack of independence of the
``observations'' that results from fitting multiple EMs to each
simulated data set.

The second method involved fitting classification and regression trees
(Breiman et al. 1984) to show how the OM and EM attributes, and their
interactions, partition the values for each measure of reliability
(e.g., Gonzalez et al. 2018; Collier et al. 2022). We used
classification trees for categorical measures (convergence and AIC) and
regression trees for the other measures with continuous scales (relative
error and Mohn's \(\rho\)). The response variables were the same as the
regressions for the deviance reduction analyses. We used the
\verb|rpart| function in the rpart package (Therneau and Atkinson 2025)
to fit trees. Full trees were determined using default settings except
that we increased the number of cross-validations to 100. For clarity,
we manually pruned the full trees to show just the primary branches.

We also provide detailed results for all measures of reliability at each
combination of OM and EM attributes in the Supplementary Materials. For
confidence intervals of probability of convergence, we used the
Clopper-Pearson exact method (Clopper and Pearson 1934; Thulin 2014).
For AIC selection of process error source we provide estimates of the
proportions of simulations where each EM type was selected. For AIC
selection of the SRR (a binary indicator for each simulated data set),
we fit logistic regressions and present resulting predicted
probabilities of correctly selecting the SRR as a function of SSB
variability (\(\log \text{SD}_{\text{SSB}}\) described above). We
estimated bias as the median of the relative errors across all
simulations for a given OM and EM combination. We constructed 95\%
confidence intervals for the median relative bias, and Mohn's \(\rho\)
using the binomial distribution approach (Thompson 1936) as in Miller
and Hyun (2018) and Stock and Miller (2021).

\section*{Results}\label{results}
\addcontentsline{toc}{section}{Results}

\subsection*{Convergence performance}\label{convergence-performance}
\addcontentsline{toc}{subsection}{Convergence performance}

For probability of convergence, the EM process error assumption was the
single attribute that resulted in the largest percent reduction in
deviance (14-28\%) for all OM process error sources other than R+S OMs
where the EM median natural mortality rate assumption (estimated or
known) explained the most residual deviance (\textgreater11\%; Table
\ref{convergence_PRD_table}). However, including interactions of OM and
EM factors also provided large reductions in residual deviance
(35-47\%), suggesting successful convergence depended on a combination
OM and EM attributes.

Classification trees for each OM process error source all had the
primary branch defined using the same attribute that provided the
largest reduction in deviance (Figure \ref{conv_class}). EMs that
assumed R+S process errors converged poorly for all OMs that were
simulated with the alternative process error assumptions (R, R+M, R+Sel,
and R+q OMs). For all trees, branches based on the OM fishing mortality
history showed better convergence when the OM included a change in
fishing pressure. Branches based on whether the Beverton-Holt SRR was
assumed or not, showed better convergence when it was not estimated and
branches based on the median natural mortality rate assumption showed
better convergence when it was treated as known. For some R+M and R+Sel
OMs, better convergence was also observed when there was lower
observation uncertainty.

When convergence is defined by a gradient threshold, the primary factor
explaining deviance reduction is the same that for Hessian-based
convergence for all OM process error sources, but there are some
differences in deviance reduction for secondary factors (Table
\ref{convergence_gradient_PRD_table}), and probability of convergence,
overall, was lower (Figure \ref{conv_gradient_class}). We found a wide
range of maximum absolute values of gradients for models that had
invertible Hessians (Figure \ref{hess_grad}). The largest value observed
for a given EM and OM combination was typically \(<10^{-3}\), but many
converged models had values greater than 1. For many OMs, EMs that
assumed the correct process error source and did not estimate median
natural mortality or the Beverton-Holt SRR produced the lowest gradient
values.

\subsection*{AIC performance}\label{aic-performance}
\addcontentsline{toc}{subsection}{AIC performance}

\subsubsection*{Process error source}\label{process-error-source}
\addcontentsline{toc}{subsubsection}{Process error source}

For AIC selection of the correct process error configuration, the
magnitude of observation and process error variation were the attributes
that resulted in the largest percent reductions in deviance across OM
process error sources other than R OMs (Table \ref{AIC_PE_PRD_table}).
Both sources of variation explained large reductions in deviance for R+S
(17-22\%) and R+Sel (8-26\%) OMs, whereas variance of process errors
provided the major reductions for R+M (\textgreater9\%) and R+q
(\textgreater13\%) OMs. Comparatively, none of the OM or EM attributes
explained particularly large reductions in deviance for R OMs, but
fishing history, whether a SRR was estimated, and whether median natural
mortality was known or estimated provided similar and the largest
reductions (approximately 5-6\%). Inclusion of second and third order
interactions, did not provide large reductions in deviance for any of
the OM process error sources.

For all OM process error sources other than R OMs, the attributes
defining the primary branches of classification trees matched those that
provided the largest reductions in deviance (Figure \ref{AIC_PE_class}).
Across all OMs, AIC was more accurate for the process error source when
process error variability was greater and when observation error was
lower. For R+S OMs, there was a tendency to select R OMs when
observation error was higher and apparent survival variation was lower
(\(\sigma_{2+} = 0.25\)), but accuracy for the process error source was
otherwise highly accurate. Larger variability of process error relative
to observation error was also required for accurate identification of
the correct correlation structure for R+M, R+q, and R+Sel OMs (Figure
\ref{pe_aic}). No branches were estimated for classification trees fit
to the R OMs, likely because accuracy was high across all simulations
(0.94), although inspection of the fine-scale results shows there is
some degradation in AIC selection when an SRR and median natural
mortality rate are estimated for R OMs with constant fishing pressure
and high observation error (Figure \ref{pe_aic}, top left).

\subsubsection*{Stock-recruit
relationship}\label{stock-recruit-relationship}
\addcontentsline{toc}{subsubsection}{Stock-recruit relationship}

Logistic regressions for AIC selection of the Beverton-Holt SRR, showed
OM fishing history and \(\log \text{SD}_{\text{SSB}}\) provided
substantial reductions in deviance for R+M (\textgreater13\%), R+Sel
(\textgreater26\%), and R+q (\textgreater24\%) OMs (Table
\ref{AIC_SRR_PRD_table}). For R OMs, fishing history provided the
largest reduction in deviance (\textgreater9\%), whereas none of the
attributes individually provided large reductions in deviance for R+S
OMs (all \textless5\%). However, inclusion of all attributes provided
larger reductions in deviance than the sum of individual contributions
for both R (\textgreater30\%) and R+S (\textasciitilde19\%) OMs. Further
fits for R and R+S OMs that including different combinations of two
factors additively showed fits that included
\(\log \text{SD}_{\text{SSB}}\) and recruitment variation only provided
essentially the same reduction in deviance as the models with all
factors. For all OM process error sources, inclusion of interaction
terms provided relatively little reduction in residual deviance.

Attributes defining the primary branches of classification trees for AIC
selection of the SRR assumption were the same as those explaining the
largest reductions in deviance for the logistic regression models
(Figure \ref{AIC_SRR_class}). All branches based on
\(\log \text{SD}_{\text{SSB}}\) showed better accuracy with larger
variability in SSB and all branches based on fishing history showed
better accuracy when there was contrast in fishing pressure. Branches
based on OM observation error or recruitment variability (R and R+S OMs)
showed better accuracy when they were lower. For R OMs, a combination of
lower recruitment variability, contrast in fishing pressure, and higher
SSB variability produced AIC accuracy over 0.8. For R+S OMs, lower
recruitment variability and observation error and higher SSB variability
produced AIC accuracy of 0.79. For R+M, R+Sel, and R+q OMs, accuracy of
0.87 to 0.94 was observed with just increased SSB variability.

\subsection*{Bias}\label{bias-1}
\addcontentsline{toc}{subsection}{Bias}

\subsubsection*{Terminal year spawning stock biomass, fishing mortality,
and
recruitment}\label{terminal-year-spawning-stock-biomass-fishing-mortality-and-recruitment}
\addcontentsline{toc}{subsubsection}{Terminal year spawning stock
biomass, fishing mortality, and recruitment}

Regression models for log-scale errors in SSB that included the various
OM and EM factors showed little reduction in deviance (\textless5\%) for
any of the factors across all OM process error sources (Table
\ref{bias_SSB_PRD_table}). The attributes producing the largest
reductions were the EM assumption for median natural mortality (known or
estimated) for R, R+M, R+Sel, and R+q OMs (1-3\%), EM process error
assumption for R+S OMs (4\%) and fishing history for all OM process
error sources (1-5\%). Including second order interactions provided the
largest reductions in residual deviance (10- 26\%). Including third
order interactions also provided further reductions for R, R+S, and R+q
OMs between 5 and 11\%.

In all regression trees, branches based on fishing history and level of
observation error generally showed less bias in SSB with contrast in
fishing and lower observation error (Figure \ref{SSB_bias_regtree}). For
scenarios where there was bias, it was generally positive
(over-estimation). For branches based on treatment of median natural
mortality rate, bias was generally less when it was known rather than
estimated. For some R+Sel and R+q OMs, less bias in SSB was shown when
the EM process error assumption was correct.

Results for bias in fishing mortality and recruitment generally matched
those for SSB, except that directions of bias for fishing mortality were
opposite to those for SSB and recruitment. Effects of individual OM and
EM factors on regression models were similarly small as measured by
reduction in deviance (Tables \ref{bias_F_PRD_table} and
\ref{bias_R_PRD_table}). Factors defining the primary branches of
regression trees were in most cases identical to those for SSB (Figures
\ref{F_bias_regtree} and \ref{R_bias_regtree}).

\subsubsection*{Stock-recruit
parameters}\label{stock-recruit-parameters}
\addcontentsline{toc}{subsubsection}{Stock-recruit parameters}

Regression models for log-scale errors of estimates of both the
Beverton-Holt \(a\) and \(b\) parameters showed none of the factors
explained large percent reductions in deviance (Table
\ref{bias_SR_pars_PRD_table}). The OM fishing history provided the
largest deviance reduction for most OM process error sources for both
parameters, but reductions were generally less than 6\%. Exceptions were
the R+Sel OMs where \(a\) and \(b\) were reduced by approximately 11\%
and 8\%, respectively, and the R+q OMs where \(b\) was reduced by 10\%.
The EM process error assumption provided similar reductions in deviance
for both parameters for R OMs. Including interactions also did not
produce important reductions in deviance.

For regression trees of log-scale errors in Beverton-Holt \(a\) and
\(b\) parameter estimates, less bias was indicated with contrast in OM
fishing pressure for all branches in trees for each OM process error
source (Figures \ref{SR_a_bias_regtree} and \ref{SR_b_bias_regtree}).
For all branches based on recruitment variability in trees for R and R+S
OMs, less bias in both \(a\) and \(b\) was observed with less
recruitment variability. For R OMs with contrast in fishing pressure and
greater recruitment variability EMs that assumed the incorrect R+M
process errors produced less bias in both \(a\) and \(b\) than other
process error assumptions. Across all combinations of OM and EM
attributes, some bias was observed for both parameters, but there was
generally less bias and(or) lower variability in estimation of the \(a\)
parameter than the \(b\) parameter (Figure \ref{SR_rel_error}).

\subsubsection*{Median natural mortality
rate}\label{median-natural-mortality-rate}
\addcontentsline{toc}{subsubsection}{Median natural mortality rate}

Fitted regression models for log-scale errors in median natural
mortality rate showed largest percent reductions in residual deviance
for R+S and R+M models (Table \ref{bias_median_M_PRD_table}). The
largest reductions for a single attribute was the EM process error
assumption (\textgreater20\%) and fishing history (\textgreater15\%) for
R+S OMs. Fishing history also provided \textgreater10\% reduction for
R+M OMs, but reductions for all factors in R, R+Sel, and R+q OMs were
relatively low (\textless6\%). Interactions of OM and EM factors also
provided substantial further reductions for R+S and R+M OMs (between 8
and 15\% for second order interactions).

Regression trees with branches based on fishing history showed less bias
in median natural mortality rate with contrast in fishing pressure and
branches based on level of observation error showed less bias with more
precise observations (Figure \ref{med_M_bias_regtree}). For R OMs,
branches based on EM process error assumption showed less bias with EMs
assuming the correct R and the incorrect R+S assumption. For R+S and R+M
OMs, branches based on EM process error showed only the correct EM
process error assumption with less bias.

\subsection*{\texorpdfstring{Mohn's
\(\rho\)}{Mohn's \textbackslash rho}}\label{mohns-rho-1}
\addcontentsline{toc}{subsection}{Mohn's \(\rho\)}

Regression models for Mohn's \(\rho\) of SSB showed little reduction in
deviance for any of the OM an EM attributes (\textless2\%; Table
\ref{mohns_rho_SSB_PRD_table}). The lack of explanatory power is also
reflected in the regression trees where median Mohn's \(\rho\) values
are near zero unless a large combinations of OM and EM conditions occur
(Figure \ref{SSB_mohns_rho_regtree}). For example, in R+S OMs, with
constant fishing pressure, high observation error, and higher apparent
survival process error, EMs that assume R+M process errors have a median
Mohn's \(\rho = -0.068\).

Similarly, poor explanatory power of the OM and EM attributes occurred
when we fit regression models for Mohn's \(\rho\) of fishing mortality
and recruitment (Tables \ref{mohns_rho_F_PRD_table} and
\ref{mohns_rho_R_PRD_table}). Regression trees for Mohn's \(\rho\) of
fishing mortality were similar to those for SSB in that median values of
Mohn's \(\rho\) were close to zero for most combinations of OM and EM
attributes (Figure \ref{F_mohns_rho_regtree}). However, we observed
median Mohn's \(\rho\) for recruitment greater than 0.1 at branches much
closer to the base of the trees with fewer interactions of the OM and EM
attributes (Figure \ref{R_mohns_rho_regtree}). These branches with
consistently large retrospective patterns were typically defined by
larger OM observation error, OM constant fishing pressure, or incorrect
EM process error configuration. Comparing regression model and
regression tree fits, attributes defining the primary branches for all
regression trees of all Mohn's \(\rho\) values (SSB, fishing mortality,
and recruitment) generally matched those that explained the largest
reductions in deviance.

\section*{Discussion}\label{discussion}
\addcontentsline{toc}{section}{Discussion}

\subsection*{Assessing convergence}\label{assessing-convergence}
\addcontentsline{toc}{subsection}{Assessing convergence}

Poor convergence was common in our results when the incorrect process
error source was assumed. Li et al. (2024) found that convergence could
be a useful diagnostic especially for separating the correct simpler
process error assumption from overly complex models. Poor convergence
often occurs when parameter estimates are at their bounds (Carvalho et
al. 2021). However, even when the Hessian is invertible for a converged
model, parameters that are poorly informed will have extremely large
variance estimates. This further inspection can lead to a more
appropriate and often more parsimonious model configuration where the
problematic parameters are not estimated. For example, process error
variance parameters in state-space models that are estimated close to 0
indicates that the random effects are estimated to have little or no
variability and removing these process errors is warranted. Our
experiments did not aim to emulate the practitioner decision process in
determining an appropriate model configurations, but evaluating the
efficacy of such a decision process when applying EMs might be important
in closed loop simulations aimed at quantifying management performance
(e.g., a management strategy evaluation).

It is common during the assessment model fitting process to check that
the maximum absolute gradient component is less than some threshold
prior to inspecting the Hessian of the optimized likelihood for
invertibility (Carvalho et al. 2021), but we found reliance on magnitude
of the gradient values for fitted models as a convergence criterion
questionable. There is no accepted standard for the gradient threshold
(e.g., Lee et al. 2011; Hurtado-Ferro et al. 2014; Rudd and Thorson
2018), but the Hessian at the optimized log-likelihood was often
invertible when the maximum absolute gradient was much larger than what
might be perceived to be a sensible threshold in some of our
simulations. Therefore, the gradient criterion could exclude models that
in fact have an invertible Hessian.

A factor affecting the convergence criteria, particularly for maximum
likelihood estimation of models with random effects, is numerical
accuracy. All optimizations performed in these simulations are of the
Laplace approximation of the marginal likelihood and, therefore,
gradients and Hessians are also with respect to this approximation (see
TMB::sdreport in the Template Model Builder package). Functionality
within the Template Model Builder package exists (i.e.,
TMB::checkConsistency) to check the validity of the Laplace
approximation and the utility of this as a diagnostic for state-space
assessment models should be explored further. Furthermore, numerical
methods are used to calculate and invert the Hessian for variance
estimation for models with random effects. Our results, along with the
potential lack of accuracy imposed by these approximations, suggest at
least investigating whether the Hessian is positive definite when the
calculated absolute gradients are not terribly large (e.g, \(< 1\)).

\subsection*{Configuring process error}\label{configuring-process-error}
\addcontentsline{toc}{subsection}{Configuring process error}

We found accuracy of marginal AIC-based selection for the correct
process error source required only low observation error for R, R+S,
R+Sel, and R+q OMs. R+M OMs further required higher process error
variability, but this also improved accuracy for the other OM process
errors sources when there was higher observation error. These results
seem consistent with Li et al. (2024). Their simulation studies
investigated models with multiple process error sources and found good
accuracy of AIC in detecting correct process error assumptions for
simulations based on two stocks (Gulf of Maine cod and Southern New
England-Mid-Atlantic yellowtail founder) that are well sampled by NEUS
bottom trawl surveys used as indices in the respective assessments and
poor accuracy for Atlantic mackerel, a semi-pelagic species that is
observed relatively poorly.

\subsection*{Stock recruitment
relationships}\label{stock-recruitment-relationships}
\addcontentsline{toc}{subsection}{Stock recruitment relationships}

Variation in SSB was the most important factor for using marginal AIC to
correctly distinguish the Beverton-Holt SRR from the null model without
an SRR. For R+M, R+Sel, and R+q OMs, the SRR was accurately detected
when the CV of SSB over the time series was at least 40 to 50\%
(\(\log \text{SD}_{\text{SSB}}\) = -0.9 to -0.7) regardless of any other
OM or EM attributes. Detection of the SRR for R and R+S OMs required
lower recruitment variability, but this lower level (\(\sigma_R = 0.5\))
was assumed for all of the other OM process error source and represents
the lower range of estimates from recent NEUS stock assessments. Our
results assumed that the EM process error configuration was correct, but
this may not be a strong limitation given the ability of AIC to
distinguish the process error source in many scenarios.

Although we did not compare models with alternative SRRs (e.g., Ricker
vs.~Beverton-Holt), we do not expect AIC to perform any better
distinguishing between relationships and may be more difficult than
distinguishing from the null model even with larger variability in SSB.
Our finding that AIC tended to choose simpler recruitment models in many
cases contrasts with the noted bias in AIC for more complex models
(Shibata 1976; Katz 1981; Kass and Raftery 1995). However, these earlier
findings apply to the much more common comparison of models that are fit
to raw and independent observations, whereas our comparisons of
state-space models account for observation error and separately estimate
process errors in latent variables.

Our results comport with those of {de Valpine and Hastings} (2002) who
found AIC could not distinguish among state-space SRRs that were fit
just to SSB and recruitment observations (i.e., not within an assessment
model). Similarly, Britten et al. (In review) found AIC could not
reliably distinguish the Beverton-Holt SRR from no SRR, nor identify
alternative environmental effects on SRR parameters. However, Miller et
al. (2016) did find AIC to prefer an SRR with environmental effects when
applied to data for the Southern New England-Mid-Atlantic yellowtail
flounder stock and AIC also selected an environmental covariate on an
SRR for the most recent stock assessment of Georges Bank yellowtail
flounder (NEFSC 2025). Both of these yellowtail flounder stocks have
large changes in stock size and the values of environmental covariates
over time. Additionally, this species is well-observed by the bottom
trawl survey that is used for an index in assessment models.

Estimation of SRR parameters was only moderately reliable in ideal
scenarios of low observation error and contrast in fishing for R+Sel and
R+M OMs with large temporal variability in process errors. Otherwise,
SRR parameter estimation was biased and(or) highly variable. We found
substantial bias in estimated SRR parameters in R and R+S OMs
particularly with high variability in recruitment and apparent survival
process errors, suggesting that practitioners should be cautious with
SRR inferences when fitted assessment models have these properties. We
only evaluated effects of SSB variability on accuracy of AIC in
identifying the SRR, but those results suggests we might find less bias
for the SRR parameters in such cases as well. Another condition that
could improve perception of bias in our simulation studies is
restricting results to fits that converged with Hessian-based standard
errors for all parameters, but Britten et al. (In review) did not find
less SRR parameter bias when restricting estimates using a
gradient-based criterion. A simulation study by Stock and Miller (2021)
examining configurations of environmental covariate effects on a
Beverton-Holt SRR for the previously mentioned, well-observed, Southern
New England-Mid-Atlantic yellowtail flounder stock found little or no
bias for the density-independent mortality parameter \(a\), but still
biased estimation of the density-dependent parameter \(b\).

\subsection*{Estimating assessment model
quantities}\label{estimating-assessment-model-quantities}
\addcontentsline{toc}{subsection}{Estimating assessment model
quantities}

As expected, bias in parameters, SSB, and other assessment output was
generally improved with lower observation error. Estimation of median
natural mortality was reliable in many OM scenarios with contrast in
fishing pressure, consistent with Hoenig et al. (2025). However, we
found poor accuracy in terminal SSB estimation when estimating median
natural mortality in many OMs when there was no contrast in fishing
pressure over time and higher observation error. Therefore, estimating
median natural mortality should be approached with caution in
state-space assessment models, particularly given its significant impact
on determination of reference point and stock status (Li et al. 2024).

\subsection*{Negligible retrospective
patterns}\label{negligible-retrospective-patterns}
\addcontentsline{toc}{subsection}{Negligible retrospective patterns}

Incorrect EM process error assumptions did not produce strong
retrospective patterns for SSB for any OMs regardless of whether median
natural mortality or an SRR was estimated, although some weak patterns
occurred when observation error was high and there was contrast in
fishing pressure. However, retrospective patterns tended to be more
variable for recruitment and were sometimes large even when the EM was
correct. Therefore, we recommend de-emphasis on inspection of patterns
for recruitment, but further research on retrospective patterns in other
assessment model parameters, management quantities such as biological
reference points, and projections may be beneficial (Brooks and Legault
2016).

The general lack of retrospective patterns with mis-specified process
errors is perhaps to be expected. Retrospective patterns are often
induced in simulation studies by rapid changes in a quantity such as
index catchability, natural mortality, or perceived catch during years
toward the end of the time series (Legault 2009; Miller and Legault
2017; Huynh et al. 2022; Breivik et al. 2023). In our simulations, the
process errors changing over time may have trends in certain
simulations, particularly when strong autocorrelation is imposed, but
the random effects have no trend on average across simulations.
Szuwalski et al. (2018) and Li et al. (2024) also found relatively small
retrospective patterns when the source of mis-specification was temporal
variation in demographic attributes. Indeed, it is common for the
flexibility provided by temporal random effects to reduce retrospective
patterns (Miller et al. 2018; Stock et al. 2021; Stock and Miller 2021),
though it does not necessarily indicate a more accurate assessment model
(Perretti et al. 2020; Li et al. 2024; Liljestrand et al. 2024). Our
results together with the existing literature seem to suggest that when
a strong retrospective pattern is observed in an assessment it is more
likely to be due to a mis-specification of a rapid shift in some model
attribute rather than whether a particular process is assumed to be
randomly varying temporally.

\subsection*{Summarization approach}\label{summarization-approach}
\addcontentsline{toc}{subsection}{Summarization approach}

We found the use of regression models and classification and regression
trees extremely useful in understanding the most important OM and EM
attributes explaining variation in the measures of reliability we
examined across all simulations. The classification and regression trees
are generally a good tool for determining the OM and EM attributes that
produce better or worse measures of reliability. However, determining
the combination of attributes that produce the best or worst measures of
reliability can be challenging using the trees alone. For example, in
the regression tree for median natural mortality rate estimates in R OMs
(Figure \ref{med_M_bias_regtree}), both of the first branches imply bias
is low regardless of OM fishing history, but when OM fishing pressure is
constant, results are much better when OM observation error is low
(median RE about -6\%) than when OM observation error is high (median RE
about 40\%). The default pruning of the trees can exclude these lower
branches. However, inspection of deviance explained by various
regression models shows the \textasciitilde9\% reduction in residual
deviance by including second order interaction of all OM and EM factors
(Table \ref{bias_median_M_PRD_table}), indicating that the interaction
of factors may be important, thereby complimenting the regression tree
analysis. Higher order interactions of some factors could also provide
reductions in deviance and, therefore, inspection of results for each
combinations of OM and EM factors, as provided in the Supplementary
Materials, can also be important.

\subsection*{Recommendations and
conclusions}\label{recommendations-and-conclusions}
\addcontentsline{toc}{subsection}{Recommendations and conclusions}

Our findings regarding model convergence suggests practitioners using
state-space models and maximum marginal likelihood for estimation should
not heavily weight the magnitude of the gradient values in determining
convergence as long as the maximum absolute value is around 1 or lower.
Instead, positive-definiteness of the Hessian of the minimized negative
log-likelihood should be evaluated.

Unfortunately, whether the practitioner includes a Beverton-Holt SRR
will often depend on biological plausibility of this particular SRR
because using AIC to determine its validity required a combination of
low recruitment variability, contrast in fishing pressure, large
variation in SSB over time, and lower observation error, which applies
to a limited number of managed stocks. Furthermore, some bias in
estimation of the SRR parameters (and MSY-based reference points should
be expected. Because bias in terminal SSB and retrospective patterns
were indifferent to whether or not the SRR was estimated, the prevalence
of bias in SRR parameter estimation, and often better convergence
without the SRR, we recommend a sensible default is to exclude an SRR
when fitting assessment models, as also suggested by Brooks (2024).

We found marginal AIC can, in many cases, accurately distinguished
models with process errors. We saw the best accuracy for models with
process errors on recruitment only (R), recruitment and apparent
survival (R+S), and recruitment and selectivity (R+Sel), especially with
lower observation error. However, AIC could also distinguish R+M and R+q
process errors when variability of those processes was greater. The R+S
assumption for process errors is common in applications of WHAM in the
NEUS and the SAM assessment framework (Nielsen and Berg 2014) in ICES,
and we can have some confidence that practitioners are correctly
arriving at this assumption over other sources of process error using
marginal AIC.

\section*{Acknowledgements}\label{acknowledgements}
\addcontentsline{toc}{section}{Acknowledgements}

This work was funded by NOAA Fisheries Northeast Fisheries Science
Center. We thank Jon Deroba, two anonymous reviewers, and the associate
editor for helpful comments on earlier versions of this manuscript that
markedly improved its clarity.

\pagebreak

\section*{References}\label{references}
\addcontentsline{toc}{section}{References}

\protect\phantomsection\label{refs}
\begin{CSLReferences}{1}{0}
\bibitem[\citeproctext]{ref-aeberhardetal18}
Aeberhard, W.H., Flemming, J.M., and Nielsen, A. 2018. Review of
{State}-{Space Models} for {Fisheries Science}. Annual Review of
Statistics and Its Application \textbf{5}(1): 215--235.
doi:\href{https://doi.org/10.1146/annurev-statistics-031017-100427}{10.1146/annurev-statistics-031017-100427}.

\bibitem[\citeproctext]{ref-augeretal16}
Auger-Méthé, M., Field, C., Albertsen, C.M., Derocher, A.E., Lewis,
M.A., Jonsen, I.D., and Mills Flemming, J. 2016. State-space models'
dirty little secrets: Even simple linear {G}aussian models can have
estimation problems. Scientific reports \textbf{6}(1): 26677.
doi:\href{https://doi.org/10.1038/srep26677}{10.1038/srep26677}.

\bibitem[\citeproctext]{ref-augeretal21}
{Auger-Méthé, M., Newman, K., Cole, D., Empacher, F., Gryba, R., King,
A.A., Leos-Barajas, V., Mills Flemming, J., Nielsen, A., Petris, G., and
others}. 2021. A guide to state--space modeling of ecological time
series. Ecological Monographs \textbf{91}(4): e01470.
doi:\href{https://doi.org/10.1002/ecm.1470}{10.1002/ecm.1470}.

\bibitem[\citeproctext]{ref-breimanetal84}
Breiman, L., Friedman, J.H., Olshen, R.A., and Stone, C.J. 1984.
Classification and regression trees. Chapman; Hall/CRC, New York, NY
USA.
doi:\href{https://doi.org/10.1201/9781315139470}{10.1201/9781315139470}.

\bibitem[\citeproctext]{ref-breiviketal23}
Breivik, O.N., Aldrin, M., Fuglebakk, E., and Nielsen, A. 2023.
Detecting significant retrospective patterns in state space fish stock
assessment. Canadian Journal of Fisheries and Aquatic Sciences
\textbf{80}(9): 1509--1518.
doi:\href{https://doi.org/10.1139/cjfas-2022-0250}{10.1139/cjfas-2022-0250}.

\bibitem[\citeproctext]{ref-brittenetalinreview}
Britten, G., Brooks, E.N., and Miller, T.J. In review. Identification
and performance of environmentally-driven stock-recruitment
relationships in state space assessment models. Canadian Journal of
Fisheries and Aquatic Sciences.

\bibitem[\citeproctext]{ref-brooks24}
Brooks, E.N. 2024. Pragmatic approaches to modeling recruitment in
fisheries stock assessment: A perspective. Fisheries Research
\textbf{270}: 106896.
doi:\href{https://doi.org/10.1016/j.fishres.2023.106896}{10.1016/j.fishres.2023.106896}.

\bibitem[\citeproctext]{ref-brookslegault16}
Brooks, E.N., and Legault, C.M. 2016. Retrospective forecasting --
evaluating performance of stock projections for {N}ew {E}ngland
groundfish stocks. Canadian Journal of Fisheries and Aquatic Sciences
\textbf{73}(6): 935--950.
doi:\href{https://doi.org/10.1139/cjfas-2015-0163}{10.1139/cjfas-2015-0163}.

\bibitem[\citeproctext]{ref-cadigan16}
Cadigan, N.G. 2016. A state-space stock assessment model for northern
cod, including under-reported catches and variable natural mortality
rates. Canadian Journal of Fisheries and Aquatic Sciences
\textbf{73}(2): 296--308.
doi:\href{https://doi.org/10.1139/cjfas-2015-0047}{10.1139/cjfas-2015-0047}.

\bibitem[\citeproctext]{ref-carvalhoetal21}
Carvalho, F., Winker, H., Courtney, D., Kapur, M., Kell, L., Cardinale,
M., Schirripa, M., Kitakado, T., Yemane, D., Piner, K.R., Maunder, M.N.,
Taylor, I., Wetzel, C.R., Doering, K., Johnson, K.F., and Methot, R.D.
2021. A cookbook for using model diagnostics in integrated stock
assessments. Fisheries Research \textbf{240}: 105959.
doi:\url{https://doi.org/10.1016/j.fishres.2021.105959}.

\bibitem[\citeproctext]{ref-clopperpearson34}
Clopper, C.J., and Pearson, E.S. 1934. The use of confidence or fiducial
limits illustrated in the case of the binomial. Biometrika
\textbf{26}(4): 404--413.
doi:\href{https://doi.org/10.1093/biomet/26.4.404}{10.1093/biomet/26.4.404}.

\bibitem[\citeproctext]{ref-collieretal22}
Collier, Z.K., Zhang, H., and Soyoye, O. 2022. Alternative methods for
interpreting {M}onte {C}arlo experiments. Communications in Statistics -
Simulation and Computation: 1--16.
doi:\href{https://doi.org/10.1080/03610918.2022.2082474}{10.1080/03610918.2022.2082474}.

\bibitem[\citeproctext]{ref-connetal10}
Conn, P.B., Williams, E.H., and Shertzer, K.W. 2010. When can we
reliably estimate the productivity of fish stocks? Canadian Journal of
Fisheries and Aquatic Sciences \textbf{67}(3): 511--523.
doi:\href{https://doi.org/10.1139/F09-194}{10.1139/F09-194}.

\bibitem[\citeproctext]{ref-croninpunt21}
Cronin-Fine, L., and Punt, A.E. 2021. Modeling time-varying selectivity
in size-structured assessment models. Fisheries Research \textbf{239}:
105927. Elsevier.

\bibitem[\citeproctext]{ref-devalpinehastings02}
{de Valpine, P., and Hastings, A.} 2002. Fitting population models
incorporating process noise and observation error. Ecological Monographs
\textbf{72}(1): 57--76.

\bibitem[\citeproctext]{ref-fischetal23}
Fisch, N., Shertzer, K., Camp, E., Maunder, M., and Ahrens, R. 2023.
Process and sampling variance within fisheries stock assessment models:
Estimability, likelihood choice, and the consequences of incorrect
specification. ICES Journal of Marine Science \textbf{80}(8):
2125--2149.
doi:\href{https://doi.org/10.1093/icesjms/fsad138}{10.1093/icesjms/fsad138}.

\bibitem[\citeproctext]{ref-fleischmanetal13}
Fleischman, S.J., Catalano, M.J., Clark, R.A., and Bernard, D.R. 2013.
An age-structured state-space stock--recruit model for {P}acific salmon
(\emph{{O}ncorhynchus spp}.). Canadian Journal of Fisheries and Aquatic
Sciences \textbf{70}(3): 401--414.
doi:\href{https://doi.org/10.1139/cjfas-2012-0112}{10.1139/cjfas-2012-0112}.

\bibitem[\citeproctext]{ref-gonzalezetal18}
Gonzalez, O., O'Rourke, H.P., Wurpts, I.C., and Grimm, K.J. 2018.
Analyzing {M}onte {C}arlo simulation studies with classification and
regression trees. Structural Equation Modeling: A Multidisciplinary
Journal \textbf{25}(3): 403--413.
doi:\href{https://doi.org/10.1080/10705511.2017.1369353}{10.1080/10705511.2017.1369353}.

\bibitem[\citeproctext]{ref-harwelletal18}
Harwell, M., Kohli, N., and Peralta-Torres, Y. 2018. A survey of
reporting practices of computer simulation studies in statistical
research. The American Statistician \textbf{72}(4): 321--327.
doi:\href{https://doi.org/10.1080/00031305.2017.1342692}{10.1080/00031305.2017.1342692}.

\bibitem[\citeproctext]{ref-hoenigetal25}
Hoenig, J.M., Hearn, W.S., Leigh, G.M., and Latour, R.J. 2025.
Principles for estimating natural mortality rate. Fisheries Research
\textbf{281}: 107195.
doi:\href{https://doi.org/10.1016/j.fishres.2024.107195}{10.1016/j.fishres.2024.107195}.

\bibitem[\citeproctext]{ref-hoyleetal22}
Hoyle, S.D., Maunder, M.N., Punt, A.E., Mace, P.M., Devine, J.A., and
A'mar, Z.T. 2022. Preface: Developing the next generation of stock
assessment software. Fisheries Research \textbf{246}: 106176.
doi:\href{https://doi.org/10.1016/j.fishres.2021.106176}{10.1016/j.fishres.2021.106176}.

\bibitem[\citeproctext]{ref-hurtadoferroetal15}
Hurtado-Ferro, F., Szuwalski, C.S., Valero, J.L., Anderson, S.C.,
Cunningham, C.J., Johnson, K.F., Licandeo, R., McGilliard, C.R.,
Monnahan, C.C., Muradian, M.L., Ono, K., Vert-Pre, K.A., Whitten, A.R.,
and Punt, A.E. 2014. Looking in the rear-view mirror: Bias and
retrospective patterns in integrated, age-structured stock assessment
models. ICES Journal of Marine Science \textbf{72}(1): 99--110.
doi:\href{https://doi.org/10.1093/icesjms/fsu198}{10.1093/icesjms/fsu198}.

\bibitem[\citeproctext]{ref-huynhetal22}
Huynh, Q.C., Legault, C.M., Hordyk, A.R., and Carruthers, T.R. 2022. A
closed-loop simulation framework and indicator approach for evaluating
impacts of retrospective patterns in stock assessments. ICES Journal of
Marine Science \textbf{79}(7): 2003--2016.
doi:\href{https://doi.org/10.1093/icesjms/fsac066}{10.1093/icesjms/fsac066}.

\bibitem[\citeproctext]{ref-johnsonetal16}
Johnson, K.F., Councill, E., Thorson, J.T., Brooks, E., Methot, R.D.,
and Punt, A.E. 2016. Can autocorrelated recruitment be estimated using
integrated assessment models and how does it affect population
forecasts? Fisheries Research \textbf{183}: 222--232.
doi:\href{https://doi.org/10.1016/j.fishres.2016.06.004}{10.1016/j.fishres.2016.06.004}.

\bibitem[\citeproctext]{ref-kapuretal24}
Kapur, M.S., Ducharme-Barth, N., Oshima, M., and Carvalho, F. 2025. Good
practices, trade-offs, and precautions for model diagnostics in
integrated stock assessments. Fisheries Research \textbf{281}: 107206.
doi:\href{https://doi.org/10.1016/j.fishres.2024.107206}{10.1016/j.fishres.2024.107206}.

\bibitem[\citeproctext]{ref-kassraftery95}
Kass, R.E., and Raftery, A.E. 1995. Bayes factors. Journal of the
American Statistical Association \textbf{90}(430): 773--795.
doi:\href{https://doi.org/10.1080/01621459.1995.10476572}{10.1080/01621459.1995.10476572}.

\bibitem[\citeproctext]{ref-katz81}
Katz, R.W. 1981. On some criteria for estimating the order of a {M}arkov
chain. Technometrics \textbf{23}(3): 243--249.
doi:\href{https://doi.org/10.1080/00401706.1981.10486293}{10.1080/00401706.1981.10486293}.

\bibitem[\citeproctext]{ref-knape08}
Knape, J. 2008. Estimability of density dependence in models of time
series data. Ecology \textbf{89}(11): 2994--3000.
doi:\href{https://doi.org/10.1890/08-0071.1}{10.1890/08-0071.1}.

\bibitem[\citeproctext]{ref-kristensenetal16}
Kristensen, K., Nielsen, A., Berg, C.W., Skaug, H., and Bell, B.M. 2016.
{TMB}: Automatic differentiation and {L}aplace approximation. Journal of
Statistical Software \textbf{70}(5): 1--21.
doi:\href{https://doi.org/10.18637/jss.v070.i05}{10.18637/jss.v070.i05}.

\bibitem[\citeproctext]{ref-leeetal11}
Lee, H.-H., Maunder, M.N., Piner, K.R., and Methot, R.D. 2011.
Estimating natural mortality within a fisheries stock assessment model:
An evaluation using simulation analysis based on twelve stock
assessments. Fisheries Research \textbf{109}(1): 89--94.
doi:\href{https://doi.org/10.1016/j.fishres.2011.01.021}{10.1016/j.fishres.2011.01.021}.

\bibitem[\citeproctext]{ref-legault09}
Legault, C.M. 2009. Report of the retrospective working group, 14-16
january 2008. US Department of Commerce Northeast Fisheries Science
Center Reference Document 09-01. US Department of Commerce Northeast
Fisheries Science Center. Woods Hole, MA.

\bibitem[\citeproctext]{ref-legaultpalmer16}
Legault, C.M., and Palmer, M.C. 2016. In what direction should the
fishing mortality target change when natural mortality increases within
an assessment? Canadian Journal of Fisheries and Aquatic Sciences
\textbf{73}(3): 349--357.
doi:\href{https://doi.org/10.1139/cjfas-2015-0232}{10.1139/cjfas-2015-0232}.

\bibitem[\citeproctext]{ref-legaultrestrepo99}
Legault, C.M., and Restrepo, V.R. 1999. A flexible forward
age-structured assessment program. Col. Vol. Sci. Pap. ICCAT
\textbf{49}(2): 246--253.

\bibitem[\citeproctext]{ref-legaultetal23}
Legault, C.M., Wiedenmann, J., Deroba, J.J., Fay, G., Miller, T.J.,
Brooks, E.N., Bell, R.J., Langan, J.A., Cournane, J.M., Jones, A.W., and
Muffley, B. 2023. Data-rich but model-resistant: An evaluation of
data-limited methods to manage fisheries with failed age-based stock
assessments. Canadian Journal of Fisheries and Aquatic Sciences
\textbf{80}(1): 27--42.
doi:\href{https://doi.org/10.1139/cjfas-2022-0045}{10.1139/cjfas-2022-0045}.

\bibitem[\citeproctext]{ref-lietal25a}
Li, C., Deroba, J.J., Berger, A.M., Goethel, D.R., Langseth, B.J.,
Schueller, A.M., and Miller, T.J. 2025a. Random effects on
numbers-at-age transitions implictly account for movement dynamics and
improve performance within a state-space stock assessment. Canadian
Journal of Fisheries and Aquatic Sciences \textbf{82}.
doi:\href{https://doi.org/10.1139/cjfas-2025-0092}{10.1139/cjfas-2025-0092}.

\bibitem[\citeproctext]{ref-lietal25}
Li, C., Deroba, J.J., Miller, T.J., Legault, C.M., and Perretti, C.
2025b. Guidance on bias-correction of log-normal random effects and
observations in state-space assessment models. Canadian Journal of
Fisheries and Aquatic Sciences.
doi:\href{https://doi.org/10.1139/cjfas-2025-0093}{10.1139/cjfas-2025-0093}.

\bibitem[\citeproctext]{ref-lietal24}
Li, C., Deroba, J.J., Miller, T.J., Legault, C.M., and Perretti, C.T.
2024. An evaluation of common stock assessment diagnostic tools for
choosing among state-space models with multiple random effects
processes. Fisheries Research \textbf{273}: 106968.
doi:\href{https://doi.org/10.1016/j.fishres.2024.106968}{10.1016/j.fishres.2024.106968}.

\bibitem[\citeproctext]{ref-liljestrandetal24}
Liljestrand, E.M., Bence, J.R., and Deroba, J.J. 2024. The effect of
process variability and data quality on performance of a state-space
stock assessment model. Fisheries Research \textbf{275}: 107023.
doi:\href{https://doi.org/10.1016/j.fishres.2024.107023}{10.1016/j.fishres.2024.107023}.

\bibitem[\citeproctext]{ref-mackinnonetal95}
MacKinnon, D.P., Warsi, G., and Dwyer, J.H. 1995. A simulation study of
mediated effect measures. Multivariate Behavioral Research
\textbf{30}(1): 41--62.
doi:\href{https://doi.org/10.1207/s15327906mbr3001/_3}{10.1207/s15327906mbr3001\textbackslash\_3}.

\bibitem[\citeproctext]{ref-magnussonhilborn07}
Magnusson, A., and Hilborn, R. 2007. What makes fisheries data
informative? Fish and Fisheries \textbf{8}(4): 337--358.
doi:\href{https://doi.org/10.1111/j.1467-2979.2007.00258.x}{10.1111/j.1467-2979.2007.00258.x}.

\bibitem[\citeproctext]{ref-methotwetzel13}
Methot, R.D., and Wetzel, C.R. 2013. Stock synthesis: A biological and
statistical framework for fish stock assessment and fishery management.
Fisheries Research \textbf{142}: 86--99.
doi:\href{https://doi.org/10.1016/j.fishres.2012.10.012}{10.1016/j.fishres.2012.10.012}.

\bibitem[\citeproctext]{ref-millerbrooks21}
Miller, T.J., and Brooks, E.N. 2021. Steepness is a slippery slope. Fish
and Fisheries \textbf{22}(3): 634--645.
doi:\href{https://doi.org/10.1111/faf.12534}{10.1111/faf.12534}.

\bibitem[\citeproctext]{ref-milleretal25}
Miller, T.J., Curti, K.L., and Hansell, A.C. 2025. Space for WHAM: A
multi-region, multi-stock generalization of the woods hole assessment
model with an application to black sea bass. Canadian Journal of
Fisheries and Aquatic Sciences \textbf{82}: 1--26.
doi:\href{https://doi.org/10.1139/cjfas-2025-0097}{10.1139/cjfas-2025-0097}.

\bibitem[\citeproctext]{ref-milleretal16}
Miller, T.J., Hare, J.A., and Alade, L. 2016. A state-space approach to
incorporating environmental effects on recruitment in an age-structured
assessment model with an application to {S}outhern {N}ew {E}ngland
yellowtail flounder. Canadian Journal of Fisheries and Aquatic Sciences
\textbf{73}(8): 1261--1270.
doi:\href{https://doi.org/10.1139/cjfas-2015-0339}{10.1139/cjfas-2015-0339}.

\bibitem[\citeproctext]{ref-millerhyun18}
Miller, T.J., and Hyun, S.-Y. 2018. Evaluating evidence for alternative
natural mortality and process error assumptions using a state-space,
age-structured assessment model. Canadian Journal of Fisheries and
Aquatic Sciences \textbf{75}(5): 691--703.
doi:\href{https://doi.org/10.1139/cjfas-2017-0035}{10.1139/cjfas-2017-0035}.

\bibitem[\citeproctext]{ref-millerlegault17}
Miller, T.J., and Legault, C.M. 2017. Statistical behavior of
retrospective patterns and their effects on estimation of stock and
harvest status. Fisheries Research \textbf{186}: 109--120.
doi:\href{https://doi.org/10.1016/j.fishres.2016.08.002}{10.1016/j.fishres.2016.08.002}.

\bibitem[\citeproctext]{ref-milleretal18}
Miller, T.J., O'Brien, L., and Fratantoni, P.S. 2018. Temporal and
environmental variation in growth and maturity and effects on management
reference points of {G}eorges {B}ank {A}tlantic cod. Canadian Journal of
Fisheries and Aquatic Sciences \textbf{75}(12): 2159--2171.
doi:\href{https://doi.org/10.1139/cjfas-2017-0124}{10.1139/cjfas-2017-0124}.

\bibitem[\citeproctext]{ref-mohn99}
Mohn, R. 1999. The retrospective problem in sequential population
analysis: An investigation using cod fishery and simulated data. ICES
Journal of Marine Science \textbf{56}(4): 473--488.
doi:\href{https://doi.org/10.1006/jmsc.1999.0481}{10.1006/jmsc.1999.0481}.

\bibitem[\citeproctext]{ref-nefsc22}
NEFSC. 2022a. Final report of the haddock research track assessment
working group. {Available} at
https://s3.us-east-1.amazonaws.com/nefmc.org/14b\_EGB\_Research\_Track\_Haddock\_WG\_Report\_DRAFT.pdf.

\bibitem[\citeproctext]{ref-nefsc22a}
NEFSC. 2022b. Report of the {A}merican plaice research track working
group. {Available} at
https://s3.us-east-1.amazonaws.com/nefmc.org/2\_American-Plaice-WG-Report.pdf.

\bibitem[\citeproctext]{ref-nefsc24}
NEFSC. 2024. Butterfish research track assessment report. US Dept Commer
Northeast Fish Sci Cent Ref Doc. 24-03; 191 p.

\bibitem[\citeproctext]{ref-nefsc25}
NEFSC. 2025. Yellowttail flounder research track working group report.
{Available} at
https://d23h0vhsm26o6d.cloudfront.net/10c.-Yellowtail-Flounder-RT-WG-Report.pdf.

\bibitem[\citeproctext]{ref-nielsenberg14}
Nielsen, A., and Berg, C.W. 2014. Estimation of time-varying selectivity
in stock assessments using state-space models. Fisheries Research
\textbf{158}: 96--101.
doi:\href{https://doi.org/10.1016/j.fishres.2014.01.014}{10.1016/j.fishres.2014.01.014}.

\bibitem[\citeproctext]{ref-pedersenberg17}
Pedersen, M.W., and Berg, C.W. 2017. A stochastic surplus production
model in continuous time. Fish and Fisheries \textbf{18}(2): 226--243.
doi:\href{https://doi.org/10.1111/faf.12174}{10.1111/faf.12174}.

\bibitem[\citeproctext]{ref-perrettietal20}
Perretti, C.T., Deroba, J.J., and Legault, C.M. 2020. Simulation testing
methods for estimating misreported catch in a state-space stock
assessment model. ICES Journal of Marine Science \textbf{77}(3):
911--920.
doi:\href{https://doi.org/10.1093/icesjms/fsaa034}{10.1093/icesjms/fsaa034}.

\bibitem[\citeproctext]{ref-polanskyetal09}
Polansky, L., De Valpine, P., Lloyd-Smith, J.O., and Getz, W.M. 2009.
Likelihood ridges and multimodality in population growth rate models.
Ecology \textbf{90}(8): 2313--2320.
doi:\href{https://doi.org/10.1890/08-1461.1}{10.1890/08-1461.1}.

\bibitem[\citeproctext]{ref-dupontaviceetal22}
Pontavice, H. du, Miller, T.J., Stock, B.C., Chen, Z., and Saba, V.S.
2022. Ocean model-based covariates improve a marine fish stock
assessment when observations are limited. ICES Journal of Marine Science
\textbf{79}(4): 1259--1273.
doi:\href{https://doi.org/10.1093/icesjms/fsac050}{10.1093/icesjms/fsac050}.

\bibitem[\citeproctext]{ref-punt23}
Punt, A.E. 2023. Those who fail to learn from history are condemned to
repeat it: A perspective on current stock assessment good practices and
the consequences of not following them. Fisheries Research \textbf{261}:
106642.
doi:\href{https://doi.org/10.1016/j.fishres.2023.106642}{10.1016/j.fishres.2023.106642}.

\bibitem[\citeproctext]{ref-puntetal14}
Punt, A.E., Hurtado-Ferro, F., and Whitten, A.R. 2014. Model selection
for selectivity in fisheries stock assessments. Fisheries Research
\textbf{158}: 124--134.
doi:\href{https://doi.org/10.1016/j.fishres.2013.06.003}{10.1016/j.fishres.2013.06.003}.

\bibitem[\citeproctext]{ref-ruddthorson18}
Rudd, M.B., and Thorson, J.T. 2018. Accounting for variable recruitment
and fishing mortality in length-based stock assessments for data-limited
fisheries. Canadian Journal of Fisheries and Aquatic Sciences
\textbf{75}(7): 1019--1035.
doi:\href{https://doi.org/10.1139/cjfas-2017-0143}{10.1139/cjfas-2017-0143}.

\bibitem[\citeproctext]{ref-shibata76}
Shibata, R. 1976. Selection of the order of an autoregressive model by
{A}kaike's information criterion. Biometrika \textbf{63}(1): 117--126.
doi:\href{https://doi.org/10.1093/biomet/63.1.117}{10.1093/biomet/63.1.117}.

\bibitem[\citeproctext]{ref-stewartmonnahan17}
Stewart, I.J., and Monnahan, C.C. 2017. Implications of process error in
selectivity for approaches to weighting compositional data in fisheries
stock assessments. Fisheries Research \textbf{192}: 126--134.
doi:\href{https://doi.org/10.1016/j.fishres.2016.06.018}{10.1016/j.fishres.2016.06.018}.

\bibitem[\citeproctext]{ref-stockmiller21}
Stock, B.C., and Miller, T.J. 2021. The {W}oods {H}ole {A}ssessment
{M}odel ({WHAM}): {A} general state-space assessment framework that
incorporates time- and age-varying processes via random effects and
links to environmental covariates. Fisheries Research \textbf{240}:
105967.
doi:\href{https://doi.org/10.1016/j.fishres.2021.105967}{10.1016/j.fishres.2021.105967}.

\bibitem[\citeproctext]{ref-stocketal21}
Stock, B.C., Xu, H., Miller, T.J., Thorson, J.T., and Nye, J.A. 2021.
{Implementing two-dimensional autocorrelation in either survival or
natural mortality improves a state-space assessment model for Southern
{N}ew {E}ngland-Mid {A}tlantic yellowtail flounder}. Fisheries Research
\textbf{237}: 105873.
doi:\href{https://doi.org/10.1016/j.fishres.2021.105873}{10.1016/j.fishres.2021.105873}.

\bibitem[\citeproctext]{ref-szuwalskietal18}
Szuwalski, C.S., Ianelli, J.N., and Punt, A.E. 2018. Reducing
retrospective patterns in stock assessment and impacts on management
performance. ICES Journal of Marine Science \textbf{75}(2): 596--609.
doi:\href{https://doi.org/10.1093/icesjms/fsx159}{10.1093/icesjms/fsx159}.

\bibitem[\citeproctext]{ref-rpart}
Therneau, T., and Atkinson, B. 2025. Rpart: {R}ecursive {P}artitioning
and {R}egression {T}rees. Available from
\url{https://github.com/bethatkinson/rpart}.

\bibitem[\citeproctext]{ref-thompson36}
Thompson, W.R. 1936. On confidence ranges for the median and other
expectation distributions for populations of unknown distribution form.
Annals of Mathematical Statistics \textbf{7}(3): 122--128.
doi:\href{https://doi.org/10.1214/aoms/1177732502}{10.1214/aoms/1177732502}.

\bibitem[\citeproctext]{ref-thorsonminto15}
Thorson, J.T., and Minto, C. 2015. Mixed effects: A unifying framework
for statistical modelling in fisheries biology. ICES Journal of Marine
Science \textbf{72}(5): 1245--1256.
doi:\href{https://doi.org/10.1093/icesjms/fsu213}{10.1093/icesjms/fsu213}.

\bibitem[\citeproctext]{ref-thulin14}
Thulin, M. 2014. {The cost of using exact confidence intervals for a
binomial proportion}. Electronic Journal of Statistics \textbf{8}(1):
817--840.
doi:\href{https://doi.org/10.1214/14-EJS909}{10.1214/14-EJS909}.

\bibitem[\citeproctext]{ref-trijouletetal20}
Trijoulet, V., Fay, G., and Miller, T.J. 2020. Performance of a
state-space multispecies model: What are the consequences of ignoring
predation and process errors in stock assessments? Journal of Applied
Ecology \textbf{57}(1): 121--135.
doi:\href{https://doi.org/10.1111/1365-2664.13515}{10.1111/1365-2664.13515}.

\bibitem[\citeproctext]{ref-wangetal17}
Wang, S., Cadigan, N.G., and Benoît, H.P. 2017. Inference about
regression parameters using highly stratified survey count data with
over-dispersion and repeated measurements. Journal of Applied Statistics
\textbf{44}(6): 1013--1030.
doi:\href{https://doi.org/10.1080/02664763.2016.1191622}{10.1080/02664763.2016.1191622}.

\bibitem[\citeproctext]{ref-wiedenmannetal19}
Wiedenmann, J., Free, C.M., and Jensen, O.P. 2019. Evaluating the
performance of data-limited methods for setting catch targets through
application to data-rich stocks: A case study using northeast {U.S.}
Fish stocks. Fisheries Research \textbf{209}(1): 129--142.
doi:\href{https://doi.org/10.1016/j.fishres.2018.09.018}{10.1016/j.fishres.2018.09.018}.

\bibitem[\citeproctext]{ref-xuetal19}
Xu, H., Thorson, J.T., Methot, R.D., and Taylor, I.G. 2019. A new
semi-parametric method for autocorrelated age- and time-varying
selectivity in age-structured assessment models. Canadian Journal of
Fisheries and Aquatic Sciences \textbf{76}(2): 268--285.
doi:\href{https://doi.org/10.1139/cjfas-2017-0446}{10.1139/cjfas-2017-0446}.

\bibitem[\citeproctext]{ref-yee08}
Yee, T.W. 2008. The {VGAM} package. R News \textbf{8}(2): 28--39.
Available from
\url{https://journal.r-project.org/articles/RN-2008-014/}.

\bibitem[\citeproctext]{ref-yee15}
Yee, T.W. 2015. Vector generalized linear and additive models: With an
implementation in {R}. Springer, New York, NY USA.
doi:\href{https://doi.org/10.1007/978-1-4939-2818-7}{10.1007/978-1-4939-2818-7}.

\end{CSLReferences}

\pagebreak

\clearpage

\begin{landscape}
\begin{figure}
\begin{center}
\includegraphics[width = 1.4\textwidth]{convergence_classification_plots}
\end{center}
\caption{Classification trees indicating primary factors determining convergence as defined by providing Hessian-based standard errors for R, R+S, R+M, R+Sel and R+q OMs. Nodes denote percent convergence (top) and number of fits (bottom)  for the corresponding subset. Lower or higher convergence rates are indicated by more red or green polygons, respectively}\label{conv_class}
\end{figure}
\end{landscape}

\begin{landscape}
\begin{figure}
\begin{center}
\includegraphics[width = 1.4\textwidth]{AIC_PE_classification_plots}
\end{center}
\caption{Classification trees indicating primary factors determining which EM process error assumption provides the lowest AIC for R+S, R+M, R+Sel and R+q OMs. Each node shows the proportion of EM process error models with lowest AIC (top) and number of observations (bottom) for the corresponding subset. Lower or higher accuracy of the process error assumption are indicated by more red or green polygons, respectively.}\label{AIC_PE_class}
\end{figure}
\end{landscape}

\begin{landscape}
\begin{figure}
\begin{center}
\includegraphics[width = 1.4\textwidth]{AIC_SRR_classification_plots}
\end{center}
\caption{Classification trees indicating primary factors determining which EM SRR assumption (none or Beverton-Holt) provides the lowest AIC for R, R+S, R+M, R+Sel and R+q OMs. All EMs assume the correct process error source. Nodes denote the percentage of EMs that assume the SRR with lowest AIC (top) and number of observations (bottom) for the corresponding subset. Lower or higher accuracy of the process error assumption are indicated by more red or green polygons, respectively.}\label{AIC_SRR_class}
\end{figure}
\end{landscape}

\begin{landscape}
\begin{figure}
\begin{center}
\includegraphics[width = 1.4\textwidth]{SSB_bias_regtree_plots}
\end{center}
\caption{Regression trees indicating primary factors determining reductions in sums of squares of errors in estimation measured by Eq. \ref{bias_regression_response} for terminal year SSB for R+S, R+M, R+Sel and R+q OMs. Each node shows the median error (top) and number of observations (bottom) for the corresponding subset. Median errors closer to or further from zero are indicated by more green or red polygons, respectively.}\label{SSB_bias_regtree}
\end{figure}
\end{landscape}

\begin{landscape}
\begin{figure}
\begin{center}
\includegraphics[width = 1.4\textwidth]{SR_a_bias_regtree_plots}
\end{center}
\caption{Regression trees indicating primary factors determining reductions in sums of squares of errors in estimation measured by Eq. \ref{bias_regression_response} for the Beverton-Holt SRR parameter $a$ for R+S, R+M, R+Sel and R+q OMs. Each node shows the median error (top) and number of observations (bottom) for the corresponding subset. Lower or higher median absolute errors of the process error assumption are indicated by more green or red polygons, respectively.}\label{SR_a_bias_regtree}
\end{figure}
\end{landscape}

\begin{landscape}
\begin{figure}
\begin{center}
\includegraphics[width = 1.4\textwidth]{SR_b_bias_regtree_plots}
\end{center}
\caption{Regression trees indicating primary factors determining reductions in sums of squares of errors in estimation measured by Eq. \ref{bias_regression_response} for the Beverton-Holt SRR parameter $b$ for R+S, R+M, R+Sel and R+q OMs. Each node shows the median error (top) and number of observations (bottom) for the corresponding subset. Lower or higher median absolute errors of the process error assumption are indicated by more green or red polygons, respectively.}\label{SR_b_bias_regtree}
\end{figure}
\end{landscape}

\begin{landscape}
\begin{figure}
\begin{center}
\includegraphics[width = 1.4\textwidth]{med_M_bias_regtree_plots}
\end{center}
\caption{Regression trees indicating primary factors determining reductions in sums of squares of errors in estimation measured by Eq. \ref{bias_regression_response} for the median natural mortality rate for R+S, R+M, R+Sel and R+q OMs. Each node shows the median error (top) and number of observations (bottom) for the corresponding subset. Lower or higher median absolute errors of the process error assumption are indicated by more green or red polygons, respectively.}\label{med_M_bias_regtree}
\end{figure}
\end{landscape}

\begin{landscape}
\begin{figure}
\begin{center}
\includegraphics[width = 1.4\textwidth]{SSB_mohns_rho_regtree_plots}
\end{center}
\caption{Regression trees indicating primary factors determining reductions in sums of squares of errors in transformed Mohn's $\rho$ (Eq. \ref{bias_regression_response}) for SSB for R+S, R+M, R+Sel and R+q OMs. Each node shows the median Mohn's $\rho$ (top) and number of observations (bottom) for the corresponding subset. Median Mohn's $\rho$ closer to or further from zero are indicated by more green or red polygons, respectively.}\label{SSB_mohns_rho_regtree}
\end{figure}
\end{landscape}

\pagebreak

\begin{table}
\caption{For each OM process error source (columns), percent reduction in deviance for logistic regression models fit to indicators of convergence (providing Hessian-based standard errors) with each OM and EM factor (rows) included individually, combined, and with second and third order interactions.}\label{convergence_PRD_table}
{\begin{center}
\begin{tabular}{lrrr}
\hline\hline
\multicolumn{1}{l}{Factor}&\multicolumn{1}{c}{R}&\multicolumn{1}{c}{R+S}&\multicolumn{1}{c}{R+M}\tabularnewline
\hline
OM $F$ history& 0.86&\textless  0.01& 0.24\tabularnewline
OM Obs. Error& 0.56& 0.03& 0.26\tabularnewline
OM $\sigma_e$& 0.66& 2.99& 0.93\tabularnewline
OM $\sigma_E$& 0.53& 2.22& 0.62\tabularnewline
OM $\rho_{E}$&\textless  0.01& 0.02&\textless  0.01\tabularnewline
OM $\beta_{E}$& 0.02&\textless  0.01&\textless  0.01\tabularnewline
EM Process Error&24.53& 9.33&23.06\tabularnewline
EM $\beta_{E}$ assumption& 0.16& 2.96& 0.51\tabularnewline
EM $M$ Assumption& 0.18& 2.83& 0.40\tabularnewline
All factors&28.92&22.67&27.34\tabularnewline
+ All Two Way&36.15&33.54&34.03\tabularnewline
+ All Three Way&37.95&36.58&35.57\tabularnewline
\hline
\end{tabular}\end{center}
}
\end{table}

\begin{table}
\caption{For each OM process error source (columns), percent reduction in deviance for multinomial logistic regression models fit to indicators of EM process error assumption with lowest AIC with each OM and EM factor (rows) included individually, combined, and with second and third order interactions.}\label{AIC_PE_PRD_table}
{\begin{center}
\begin{tabular}{lrrrrr}
\hline\hline
\multicolumn{1}{l}{Factor}&\multicolumn{1}{c}{R}&\multicolumn{1}{c}{R+S}&\multicolumn{1}{c}{R+M}&\multicolumn{1}{c}{R+Sel}&\multicolumn{1}{c}{R+q}\tabularnewline
\hline
EM Assumption&11.24& 2.22& 1.69& 1.63& 3.40\tabularnewline
OM Obs. Error& 2.96&22.46& 3.42&25.67& 5.03\tabularnewline
OM $F$ History& 5.77& 0.62& 0.94& 0.91& 2.05\tabularnewline
OM $\sigma_R$& 0.10& 0.66&--&--&--\tabularnewline
OM $\sigma_{2+}$ &--&16.86&--&--&--\tabularnewline
OM $\sigma_M$&--&--& 9.06&--&--\tabularnewline
OM $\rho_R$&--&--& 0.38&--&--\tabularnewline
OM $\sigma_{Sel}$&--&--&--& 7.59&--\tabularnewline
OM $\rho_{Sel}$&--&--&--& 0.60&--\tabularnewline
OM $\sigma_q$&--&--&--&--&13.50\tabularnewline
OM $\rho_q$&--&--&--&--& 0.75\tabularnewline
All factors&20.98&46.52&16.61&40.94&26.08\tabularnewline
+ All Two Way&22.04&49.25&21.71&44.14&30.35\tabularnewline
+ All Three Way&22.07&49.99&22.41&44.58&31.47\tabularnewline
\hline
\end{tabular}\end{center}
}
\end{table}

\begin{table}
\caption{For each OM process error source (columns), percent reduction in deviance for logistic regression models fit to indicators of EM SRR assumption (none or Beverton-Holt) with lowest AIC with each OM and EM factor (rows) included individually, combined, and with second and third order interactions.}\label{AIC_SRR_PRD_table}
{\begin{center}
\begin{tabular}{lrrrrr}
\hline\hline
\multicolumn{1}{l}{Factor}&\multicolumn{1}{c}{R}&\multicolumn{1}{c}{R+S}&\multicolumn{1}{c}{R+M}&\multicolumn{1}{c}{R+Sel}&\multicolumn{1}{c}{R+q}\tabularnewline
\hline
EM $M$ Assumption& 0.04& 0.21& 0.18& 0.02& 0.01\tabularnewline
OM Obs. Error&\textless  0.01& 0.65& 0.14& 0.04& 0.02\tabularnewline
OM $F$ History& 9.17& 3.79&13.08&26.56&24.60\tabularnewline
OM $\sigma_R$& 3.54& 4.74&--&--&--\tabularnewline
OM $\sigma_{2+}$ &--& 0.14&--&--&--\tabularnewline
OM $\sigma_M$&--&--& 1.14&--&--\tabularnewline
OM $\rho_R$&--&--& 0.05&--&--\tabularnewline
OM $\sigma_{Sel}$&--&--&--& 0.02&--\tabularnewline
OM $\rho_{Sel}$&--&--&--& 0.17&--\tabularnewline
OM $\sigma_q$&--&--&--&--& 0.36\tabularnewline
OM $\rho_q$&--&--&--&--& 0.02\tabularnewline
$\log\left(\text{SD}_\text{SSB}\right)$& 4.11& 1.59&33.39&41.36&39.23\tabularnewline
All factors&31.52&18.99&34.23&43.77&42.31\tabularnewline
+ All Two Way&34.79&22.24&35.99&45.84&44.04\tabularnewline
+ All Three Way&35.41&23.09&37.57&46.39&44.63\tabularnewline
\hline
\end{tabular}\end{center}
}
\end{table}

\begin{table}
\caption{For each OM process error source (columns), percent reduction in deviance for linear regression models fit to errors in estimation measured by Eq. \ref{bias_regression_response} for the terminal year SSB with each OM and EM factor (rows) included individually, combined, and with second and third order interactions.}\label{bias_SSB_PRD_table}
{\begin{center}
\begin{tabular}{lrrr}
\hline\hline
\multicolumn{1}{l}{Factor}&\multicolumn{1}{c}{R}&\multicolumn{1}{c}{R+S}&\multicolumn{1}{c}{R+M}\tabularnewline
\hline
Convergence& 0.02& 0.13&\textless  0.01\tabularnewline
OM $F$ history& 2.08& 2.74& 1.75\tabularnewline
OM Obs. Error& 1.09& 0.18& 1.19\tabularnewline
$OM \sigma_e$& 0.01&\textless  0.01&\textless  0.01\tabularnewline
$OM \sigma_E$&\textless  0.01&\textless  0.01&\textless  0.01\tabularnewline
$OM \rho_{E}$&\textless  0.01&\textless  0.01&\textless  0.01\tabularnewline
OM $\beta_{E}$&\textless  0.01& 0.01&\textless  0.01\tabularnewline
EM Process Error& 0.52& 1.05& 0.57\tabularnewline
$EM \beta_{E}$ assumption&\textless  0.01& 0.05& 0.01\tabularnewline
EM $M$ assumption& 1.76& 2.12& 1.51\tabularnewline
All factors& 5.97& 6.58& 5.21\tabularnewline
+ All Two Way&12.81&13.00&11.97\tabularnewline
+ All Three Way&16.41&15.57&15.73\tabularnewline
\hline
\end{tabular}\end{center}
}
\end{table}

\begin{landscape}
\begin{table}
\caption{For each OM process error source (columns), percent reduction in deviance for linear regression models fit to errors in estimation measured by Eq. \ref{bias_regression_response} for the Beverton-Holt SRR parameters with each OM and EM factor (rows) included individually, combined, and with second and third order interactions.}\label{bias_SR_pars_PRD_table}
{\begin{center}
\begin{tabular}{lrrrrrcrrrrr}
\hline\hline
\multicolumn{1}{l}{\bfseries Factor}&\multicolumn{5}{c}{\bfseries Beverton-Holt $a$}&\multicolumn{1}{c}{\bfseries }&\multicolumn{5}{c}{\bfseries Beverton-Holt $b$}\tabularnewline
\cline{2-6} \cline{8-12}
\multicolumn{1}{l}{}&\multicolumn{1}{c}{R}&\multicolumn{1}{c}{R+S}&\multicolumn{1}{c}{R+M}&\multicolumn{1}{c}{R+Sel}&\multicolumn{1}{c}{R+q}&\multicolumn{1}{c}{}&\multicolumn{1}{c}{R}&\multicolumn{1}{c}{R+S}&\multicolumn{1}{c}{R+M}&\multicolumn{1}{c}{R+Sel}&\multicolumn{1}{c}{R+q}\tabularnewline
\hline
EM $M$ Assumption& 0.01& 0.29& 0.08& 0.05& 0.02&& 0.18& 0.56&\textless  0.01& 0.17& 0.05\tabularnewline
EM Process Error& 2.21& 0.24& 0.81& 1.40& 1.36&& 1.92& 0.12& 0.50& 1.27& 0.53\tabularnewline
OM Obs. Error& 0.17& 0.01& 0.04& 0.13& 0.10&& 0.49&\textless  0.01& 0.11& 0.13& 0.05\tabularnewline
OM $F$ History& 8.45& 6.66&10.58&22.29&16.14&& 9.56& 6.52& 7.32&21.21&17.99\tabularnewline
OM $\sigma_R$& 3.68& 1.99&--&--&--&& 3.77& 1.73&--&--&--\tabularnewline
OM $\sigma_{2+}$ &--& 0.40&--&--&--&&--& 0.34&--&--&--\tabularnewline
OM $\sigma_M$&--&--& 0.06&--&--&&--&--& 0.02&--&--\tabularnewline
OM $\rho_R$&--&--& 0.34&--&--&&--&--& 0.69&--&--\tabularnewline
OM $\sigma_{Sel}$&--&--&--& 0.02&--&&--&--&--& 0.06&--\tabularnewline
OM $\rho_{Sel}$&--&--&--& 0.01&--&&--&--&--& 0.04&--\tabularnewline
OM $\sigma_q$&--&--&--&--& 0.17&&--&--&--&--& 0.71\tabularnewline
OM $\rho_q$&--&--&--&--& 0.08&&--&--&--&--& 0.03\tabularnewline
All factors&13.72& 9.16&11.83&23.28&17.52&&15.33& 8.77& 8.57&22.32&19.21\tabularnewline
+ All Two Way&15.10&11.20&14.08&23.96&18.95&&16.74&11.41&10.22&23.16&20.14\tabularnewline
+ All Three Way&15.76&11.77&15.14&24.84&19.92&&17.35&11.99&11.12&24.10&21.25\tabularnewline
\hline
\end{tabular}\end{center}
}
\end{table}
\end{landscape}

\begin{table}
\caption{For each OM process error source (columns), percent reduction in deviance for linear regression models fit to errors in estimation measured by Eq. \ref{bias_regression_response} for the median natural mortality rate parameter with each OM and EM factor (rows) included individually, combined, and with second and third order interactions.}\label{bias_median_M_PRD_table}
{\begin{center}
\begin{tabular}{lrrrrr}
\hline\hline
\multicolumn{1}{l}{Factor}&\multicolumn{1}{c}{R}&\multicolumn{1}{c}{R+S}&\multicolumn{1}{c}{R+M}&\multicolumn{1}{c}{R+Sel}&\multicolumn{1}{c}{R+q}\tabularnewline
\hline
EM SR assumption& 0.17& 0.59& 0.21& 0.23& 0.54\tabularnewline
EM Process Error& 0.80&18.13& 1.93& 1.69& 0.75\tabularnewline
OM Obs. Error&14.81& 0.14& 4.59& 6.17& 5.93\tabularnewline
OM $F$ History&25.11&22.21&23.18&20.70&21.95\tabularnewline
OM $\sigma_R$& 0.14& 0.05&--&--&--\tabularnewline
OM $\sigma_{2+}$ &--& 6.01&--&--&--\tabularnewline
OM $\sigma_M$&--&--& 8.62&--&--\tabularnewline
OM $\rho_R$&--&--& 1.44&--&--\tabularnewline
OM $\sigma_{Sel}$&--&--&--& 0.03&--\tabularnewline
OM $\rho_{Sel}$&--&--&--& 0.15&--\tabularnewline
OM $\sigma_q$&--&--&--&--& 2.58\tabularnewline
OM $\rho_q$&--&--&--&--& 0.15\tabularnewline
All factors&43.09&45.24&40.27&29.86&31.99\tabularnewline
+ All Two Way&43.87&54.59&47.84&36.23&35.47\tabularnewline
+ All Three Way&44.15&56.53&50.74&37.88&37.27\tabularnewline
\hline
\end{tabular}\end{center}
}
\end{table}

\begin{table}
\caption{For each OM process error source (columns), percent reduction in deviance for linear regression models fit to transformed Mohn's $\rho$ values for each simulation (Eq. \ref{bias_regression_response}) for SSB with each OM and EM factor (rows) included individually, combined, and with second and third order interactions.}\label{mohns_rho_SSB_PRD_table}
{\begin{center}
\begin{tabular}{lrrrrr}
\hline\hline
\multicolumn{1}{l}{Factor}&\multicolumn{1}{c}{R}&\multicolumn{1}{c}{R+S}&\multicolumn{1}{c}{R+M}&\multicolumn{1}{c}{R+Sel}&\multicolumn{1}{c}{R+q}\tabularnewline
\hline
EM $M$ Assumption&0.79&0.18&0.15&0.95&1.24\tabularnewline
EM SR assumption&\textless  0.01&0.01&\textless  0.01&\textless  0.01&\textless  0.01\tabularnewline
EM Process Error&\textless  0.01&0.22&0.14&0.08&0.04\tabularnewline
OM Obs. Error&0.12&0.03&0.05&0.18&0.21\tabularnewline
OM $F$ History&0.84&0.14&0.07&1.08&1.56\tabularnewline
OM $\sigma_R$&0.01&0.01&--&--&--\tabularnewline
OM $\sigma_{2+}$ &--&0.02&--&--&--\tabularnewline
OM $\sigma_M$&--&--&0.01&--&--\tabularnewline
OM $\rho_M$&--&--&\textless  0.01&--&--\tabularnewline
OM $\sigma_{\text{Sel}}$&--&--&--&0.01&--\tabularnewline
OM $\rho_{\text{Sel}}$&--&--&--&0.02&--\tabularnewline
OM $\sigma_q$&--&--&--&--&0.01\tabularnewline
OM $\rho_q$&--&--&--&--&0.01\tabularnewline
All factors&1.89&0.63&0.43&2.43&3.29\tabularnewline
+ All Two Way&3.63&1.10&0.91&4.75&6.22\tabularnewline
+ All Three Way&4.27&1.65&1.50&5.73&7.53\tabularnewline
\hline
\end{tabular}\end{center}
}
\end{table}

\begin{landscape}
\end{landscape}
\pagebreak

\setcounter{figure}{0}
\renewcommand\thefigure{S\arabic{figure}}

\setcounter{table}{0}
\renewcommand\thetable{S\arabic{table}}

\pagebreak

\section*{Supplementary Materials}\label{supplementary-materials}
\addcontentsline{toc}{section}{Supplementary Materials}

\pagebreak

\subsection*{Referenced Figures}\label{referenced-figures}
\addcontentsline{toc}{subsection}{Referenced Figures}

\begin{figure}[!ht]
\begin{center}
\includegraphics[width = \textwidth]{om_input_plots_figure}
\end{center}
\caption{The proportion mature at age, weight at age, fleet and index selectivity at age, and Beverton-Holt SRR assumed for the population in all OMs. For OMs with random effects on fleet selectivity, this represents the selectivity at the mean of the random effects.}\label{om_inputs_fig}
\end{figure}

\begin{landscape}
\begin{figure}
\begin{center}
\includegraphics[width = 1.4\textwidth]{convergence_gradient_classification_plots}
\end{center}
\caption{Classification trees indicating primary factors determining convergence as defined by a maximum absolute gradient < $10^{-6}$ for R, R+S, R+M, R+Sel and R+q OMs. Nodes denote percent convergence (top) and number of fits (bottom)  for the corresponding subset. Lower or higher convergence rates are indicated by more red or green polygons, respectively}\label{conv_gradient_class}
\end{figure}
\end{landscape}

\begin{landscape}
\begin{figure}
\begin{center}
\includegraphics[width = 1.4\textwidth]{hess_grad_convergence_plots}
\end{center}
\caption{The maximum of the absolute values of all gradient values for all fits that provided Hessian-based standard errors across all simuated data sets of a given OM configuration (A: R and R+S, B: R+M, C: R+Sel, or D: R+q).  Results are conditional on EM fits with alternative process error assumptions (colored points and lines), median natural mortality (estimated or known) and recruitment assumptions (Beverton-Holt SRR or not). Circled values indicate results where the EM process error structure matches that of the OM and vertical lines represent 95\% confidence intervals.}\label{hess_grad}
\end{figure}
\end{landscape}

\begin{landscape}
\begin{figure}
\begin{center}
\includegraphics[width = 1.4\textwidth]{pe_aic_plots}
\end{center}
\caption{Estimated probability of lowest AIC for EMs assuming alternative process error assumptions (colored bars) conditional on alternative assumptions for median natural mortality (estimated or known) and Beverton-Holt SRR (estimated or not; along x-axis) when fitted to OMs that have R and R+S (A), R+Sel (B), R+M (C), or R+q (D) process error sources. Striped bars indicate results where the EM process error structure matches that of the OM.}\label{pe_aic}
\end{figure}
\end{landscape}

\begin{landscape}
\begin{figure}
\begin{center}
\includegraphics[width = 1.4\textwidth]{F_bias_regtree_plots}
\end{center}
\caption{Regression trees indicating primary factors determining reductions in sums of squares of errors in estimation measured by Eq. \ref{bias_regression_response} for terminal year fully-selected fishing mortality for R+S, R+M, R+Sel and R+q OMs. Each node shows the median error (top) and number of observations (bottom) for the corresponding subset. Median errors closer to or further from zero are indicated by more green or red polygons, respectively.}\label{F_bias_regtree}
\end{figure}
\end{landscape}

\begin{landscape}
\begin{figure}
\begin{center}
\includegraphics[width = 1.4\textwidth]{R_bias_regtree_plots}
\end{center}
\caption{Regression trees indicating primary factors determining reductions in sums of squares of errors in estimation measured by Eq. \ref{bias_regression_response} for terminal year recruitment for R+S, R+M, R+Sel and R+q OMs. Each node shows the median error (top) and number of observations (bottom) for the corresponding subset. Median errors closer to or further from zero are indicated by more green or red polygons, respectively.}\label{R_bias_regtree}
\end{figure}
\end{landscape}

\begin{landscape}
\begin{figure}
\begin{center}
\includegraphics[width = 1.4\textwidth]{sr_bias_plots}
\end{center}
\caption{Median relative error of Beverton-Holt SRR parameters ($a$ and $b$) for EMs fitted to data sets simulated with alternative process error structures: R and R+S (A), R+Sel (B), R+M (C), or R+q (D). Circled values indicate results where the EM process error structure matches that of the OM and vertical lines represent 95\% confidence intervals.}\label{SR_rel_error}
\end{figure}
\end{landscape}

\begin{landscape}
\begin{figure}
\begin{center}
\includegraphics[width = 1.4\textwidth]{F_mohns_rho_regtree_plots}
\end{center}
\caption{Regression trees indicating primary factors determining reductions in sums of squares of errors in transformed Mohn's $\rho$ (Eq. \ref{bias_regression_response}) for fishing mortality averaged over all age classes for R+S, R+M, R+Sel and R+q OMs. Each node shows the median Mohn's $\rho$ (top) and number of observations (bottom) for the corresponding subset. Median Mohn's $\rho$ closer to or further from zero are indicated by more green or red polygons, respectively.}\label{F_mohns_rho_regtree}
\end{figure}
\end{landscape}

\begin{landscape}
\begin{figure}
\begin{center}
\includegraphics[width = 1.4\textwidth]{R_mohns_rho_regtree_plots}
\end{center}
\caption{Regression trees indicating primary factors determining reductions in sums of squares of errors in transformed Mohn's $\rho$ (Eq. \ref{bias_regression_response}) for recruitment for R+S, R+M, R+Sel and R+q OMs. Each node shows the median Mohn's $\rho$ (top) and number of observations (bottom) for the corresponding subset. Median Mohn's $\rho$ closer to or further from zero are indicated by more green or red polygons, respectively.}\label{R_mohns_rho_regtree}
\end{figure}
\end{landscape}

\subsection*{Referenced Tables}\label{referenced-tables}
\addcontentsline{toc}{subsection}{Referenced Tables}

\begin{table}
\caption{Distinguishing characteristics of the OMs with random effects on recruitment and apparent survival (R, R+S). When observation uncertainty is low, standard deviations for log-normal distributed indices and logistic normal distributed age composition observations are 0.1 and 0.3, respectively, and when it is high, standard deviations are 0.4 and 1.5, respectively. Fishing mortality either changes from 2.5$F_{\text{MSY}}$ to $F_{\text{MSY}}$ after year 20 (of 40) or is constant at $F_{\text{MSY}}$ over all years.}\label{naa_om_table}
{\footnotesize \begin{center}
\begin{tabular}{rrrrr}
\hline\hline
\multicolumn{1}{c}{Model}&\multicolumn{1}{c}{$\sigma_R$}&\multicolumn{1}{c}{$\sigma_{2+}$}&\multicolumn{1}{c}{Fishing History}&\multicolumn{1}{c}{Observation Uncertainty}\tabularnewline
\hline
NAA_1&$0.5$&$$&$2.5 F_{\text{MSY}} \rightarrow F_{\text{MSY}}$&Index sigma (log scale) = 0.1, Age composition sigma (logistic normal) = 0.3\tabularnewline
NAA_2&$1.5$&$$&$2.5 F_{\text{MSY}} \rightarrow F_{\text{MSY}}$&Index sigma (log scale) = 0.1, Age composition sigma (logistic normal) = 0.3\tabularnewline
NAA_3&$0.5$&$0.25$&$2.5 F_{\text{MSY}} \rightarrow F_{\text{MSY}}$&Index sigma (log scale) = 0.1, Age composition sigma (logistic normal) = 0.3\tabularnewline
NAA_4&$1.5$&$0.25$&$2.5 F_{\text{MSY}} \rightarrow F_{\text{MSY}}$&Index sigma (log scale) = 0.1, Age composition sigma (logistic normal) = 0.3\tabularnewline
NAA_5&$0.5$&$0.50$&$2.5 F_{\text{MSY}} \rightarrow F_{\text{MSY}}$&Index sigma (log scale) = 0.1, Age composition sigma (logistic normal) = 0.3\tabularnewline
NAA_6&$1.5$&$0.50$&$2.5 F_{\text{MSY}} \rightarrow F_{\text{MSY}}$&Index sigma (log scale) = 0.1, Age composition sigma (logistic normal) = 0.3\tabularnewline
NAA_7&$0.5$&$$&F_{\text{MSY}}&Index sigma (log scale) = 0.1, Age composition sigma (logistic normal) = 0.3\tabularnewline
NAA_8&$1.5$&$$&F_{\text{MSY}}&Index sigma (log scale) = 0.1, Age composition sigma (logistic normal) = 0.3\tabularnewline
NAA_9&$0.5$&$0.25$&F_{\text{MSY}}&Index sigma (log scale) = 0.1, Age composition sigma (logistic normal) = 0.3\tabularnewline
NAA_10&$1.5$&$0.25$&F_{\text{MSY}}&Index sigma (log scale) = 0.1, Age composition sigma (logistic normal) = 0.3\tabularnewline
NAA_11&$0.5$&$0.50$&F_{\text{MSY}}&Index sigma (log scale) = 0.1, Age composition sigma (logistic normal) = 0.3\tabularnewline
NAA_12&$1.5$&$0.50$&F_{\text{MSY}}&Index sigma (log scale) = 0.1, Age composition sigma (logistic normal) = 0.3\tabularnewline
NAA_13&$0.5$&$$&$2.5 F_{\text{MSY}} \rightarrow F_{\text{MSY}}$&Index sigma (log scale) = 0.4, Age composition sigma (logistic normal) = 1.5\tabularnewline
NAA_14&$1.5$&$$&$2.5 F_{\text{MSY}} \rightarrow F_{\text{MSY}}$&Index sigma (log scale) = 0.4, Age composition sigma (logistic normal) = 1.5\tabularnewline
NAA_15&$0.5$&$0.25$&$2.5 F_{\text{MSY}} \rightarrow F_{\text{MSY}}$&Index sigma (log scale) = 0.4, Age composition sigma (logistic normal) = 1.5\tabularnewline
NAA_16&$1.5$&$0.25$&$2.5 F_{\text{MSY}} \rightarrow F_{\text{MSY}}$&Index sigma (log scale) = 0.4, Age composition sigma (logistic normal) = 1.5\tabularnewline
NAA_17&$0.5$&$0.50$&$2.5 F_{\text{MSY}} \rightarrow F_{\text{MSY}}$&Index sigma (log scale) = 0.4, Age composition sigma (logistic normal) = 1.5\tabularnewline
NAA_18&$1.5$&$0.50$&$2.5 F_{\text{MSY}} \rightarrow F_{\text{MSY}}$&Index sigma (log scale) = 0.4, Age composition sigma (logistic normal) = 1.5\tabularnewline
NAA_19&$0.5$&$$&F_{\text{MSY}}&Index sigma (log scale) = 0.4, Age composition sigma (logistic normal) = 1.5\tabularnewline
NAA_20&$1.5$&$$&F_{\text{MSY}}&Index sigma (log scale) = 0.4, Age composition sigma (logistic normal) = 1.5\tabularnewline
NAA_21&$0.5$&$0.25$&F_{\text{MSY}}&Index sigma (log scale) = 0.4, Age composition sigma (logistic normal) = 1.5\tabularnewline
NAA_22&$1.5$&$0.25$&F_{\text{MSY}}&Index sigma (log scale) = 0.4, Age composition sigma (logistic normal) = 1.5\tabularnewline
NAA_23&$0.5$&$0.50$&F_{\text{MSY}}&Index sigma (log scale) = 0.4, Age composition sigma (logistic normal) = 1.5\tabularnewline
NAA_24&$1.5$&$0.50$&F_{\text{MSY}}&Index sigma (log scale) = 0.4, Age composition sigma (logistic normal) = 1.5\tabularnewline
\hline
\end{tabular}\end{center}
}
\end{table}

\begin{table}
\caption{Distinguishing characteristics of the OMs with random effects on recruitment and natural mortality (R+M). When observation uncertainty is low, standard deviations for log-normal distributed indices and logistic normal distributed age composition observations are 0.1 and 0.3, respectively, and when it is high, standard deviations are 0.4 and 1.5, respectively. Fishing mortality either changes from 2.5$F_{\text{MSY}}$ to $F_{\text{MSY}}$ after year 20 (of 40) or is constant at $F_{\text{MSY}}$ over all years. For AR1 process errors, $\sigma_M$ is defined for the marginal distribution of the processes.}\label{M_om_table}
{\begin{center}
\begin{tabular}{rrrrrr}
\hline\hline
\multicolumn{1}{c}{Model}&\multicolumn{1}{c}{$\sigma_R$}&\multicolumn{1}{c}{$\rho_{M}$}&\multicolumn{1}{c}{Fishing History}&\multicolumn{1}{c}{Observation Uncertainty}&\multicolumn{1}{c}{obs_error}\tabularnewline
\hline
M_1&$0.5$&$0.1$&$0.0$&$2.5 F_{\text{MSY}} \rightarrow F_{\text{MSY}}$&Index sigma (log scale) = 0.1, Age composition sigma (logistic normal) = 0.3\tabularnewline
M_2&$0.5$&$0.5$&$0.0$&$2.5 F_{\text{MSY}} \rightarrow F_{\text{MSY}}$&Index sigma (log scale) = 0.1, Age composition sigma (logistic normal) = 0.3\tabularnewline
M_3&$0.5$&$0.1$&$0.9$&$2.5 F_{\text{MSY}} \rightarrow F_{\text{MSY}}$&Index sigma (log scale) = 0.1, Age composition sigma (logistic normal) = 0.3\tabularnewline
M_4&$0.5$&$0.5$&$0.9$&$2.5 F_{\text{MSY}} \rightarrow F_{\text{MSY}}$&Index sigma (log scale) = 0.1, Age composition sigma (logistic normal) = 0.3\tabularnewline
M_5&$0.5$&$0.1$&$0.0$&F_{\text{MSY}}&Index sigma (log scale) = 0.1, Age composition sigma (logistic normal) = 0.3\tabularnewline
M_6&$0.5$&$0.5$&$0.0$&F_{\text{MSY}}&Index sigma (log scale) = 0.1, Age composition sigma (logistic normal) = 0.3\tabularnewline
M_7&$0.5$&$0.1$&$0.9$&F_{\text{MSY}}&Index sigma (log scale) = 0.1, Age composition sigma (logistic normal) = 0.3\tabularnewline
M_8&$0.5$&$0.5$&$0.9$&F_{\text{MSY}}&Index sigma (log scale) = 0.1, Age composition sigma (logistic normal) = 0.3\tabularnewline
M_9&$0.5$&$0.1$&$0.0$&$2.5 F_{\text{MSY}} \rightarrow F_{\text{MSY}}$&Index sigma (log scale) = 0.4, Age composition sigma (logistic normal) = 1.5\tabularnewline
M_10&$0.5$&$0.5$&$0.0$&$2.5 F_{\text{MSY}} \rightarrow F_{\text{MSY}}$&Index sigma (log scale) = 0.4, Age composition sigma (logistic normal) = 1.5\tabularnewline
M_11&$0.5$&$0.1$&$0.9$&$2.5 F_{\text{MSY}} \rightarrow F_{\text{MSY}}$&Index sigma (log scale) = 0.4, Age composition sigma (logistic normal) = 1.5\tabularnewline
M_12&$0.5$&$0.5$&$0.9$&$2.5 F_{\text{MSY}} \rightarrow F_{\text{MSY}}$&Index sigma (log scale) = 0.4, Age composition sigma (logistic normal) = 1.5\tabularnewline
M_13&$0.5$&$0.1$&$0.0$&F_{\text{MSY}}&Index sigma (log scale) = 0.4, Age composition sigma (logistic normal) = 1.5\tabularnewline
M_14&$0.5$&$0.5$&$0.0$&F_{\text{MSY}}&Index sigma (log scale) = 0.4, Age composition sigma (logistic normal) = 1.5\tabularnewline
M_15&$0.5$&$0.1$&$0.9$&F_{\text{MSY}}&Index sigma (log scale) = 0.4, Age composition sigma (logistic normal) = 1.5\tabularnewline
M_16&$0.5$&$0.5$&$0.9$&F_{\text{MSY}}&Index sigma (log scale) = 0.4, Age composition sigma (logistic normal) = 1.5\tabularnewline
\hline
\end{tabular}\end{center}
}
\end{table}

\begin{table}
\caption{Distinguishing characteristics of the OMs with random effects on recruitment and selectivity (R+Sel). When observation uncertainty is low, standard deviations for log-normal distributed indices and logistic normal distributed age composition observations are 0.1 and 0.3, respectively, and when it is high, standard deviations are 0.4 and 1.5, respectively. Fishing mortality either changes from 2.5$F_{\text{MSY}}$ to $F_{\text{MSY}}$ after year 20 (of 40) or is constant at $F_{\text{MSY}}$ over all years. For AR1 process errors, $\sigma_{\text{Sel}}$ is defined for the marginal distribution of the processes.}\label{sel_om_table}
{\begin{center}
\begin{tabular}{rrrrrr}
\hline\hline
\multicolumn{1}{c}{Model}&\multicolumn{1}{c}{$\sigma_R$}&\multicolumn{1}{c}{$\sigma_{\text{Sel}}$}&\multicolumn{1}{c}{$\rho_{\text{Sel}}$}&\multicolumn{1}{c}{Fishing History}&\multicolumn{1}{c}{Observation Uncertainty}\tabularnewline
\hline
$\text{Sel}_{1}$&$0.5$&$0.1$&$0.0$&$2.5 F_{\text{MSY}} \rightarrow F_{\text{MSY}}$&Index SD = 0.1, Age composition SD = 0.3\tabularnewline
$\text{Sel}_{2}$&$0.5$&$0.5$&$0.0$&$2.5 F_{\text{MSY}} \rightarrow F_{\text{MSY}}$&Index SD = 0.1, Age composition SD = 0.3\tabularnewline
$\text{Sel}_{3}$&$0.5$&$0.1$&$0.9$&$2.5 F_{\text{MSY}} \rightarrow F_{\text{MSY}}$&Index SD = 0.1, Age composition SD = 0.3\tabularnewline
$\text{Sel}_{4}$&$0.5$&$0.5$&$0.9$&$2.5 F_{\text{MSY}} \rightarrow F_{\text{MSY}}$&Index SD = 0.1, Age composition SD = 0.3\tabularnewline
$\text{Sel}_{5}$&$0.5$&$0.1$&$0.0$&$F_{\text{MSY}}$&Index SD = 0.1, Age composition SD = 0.3\tabularnewline
$\text{Sel}_{6}$&$0.5$&$0.5$&$0.0$&$F_{\text{MSY}}$&Index SD = 0.1, Age composition SD = 0.3\tabularnewline
$\text{Sel}_{7}$&$0.5$&$0.1$&$0.9$&$F_{\text{MSY}}$&Index SD = 0.1, Age composition SD = 0.3\tabularnewline
$\text{Sel}_{8}$&$0.5$&$0.5$&$0.9$&$F_{\text{MSY}}$&Index SD = 0.1, Age composition SD = 0.3\tabularnewline
$\text{Sel}_{9}$&$0.5$&$0.1$&$0.0$&$2.5 F_{\text{MSY}} \rightarrow F_{\text{MSY}}$&Index SD = 0.4, Age composition SD = 1.5\tabularnewline
$\text{Sel}_{10}$&$0.5$&$0.5$&$0.0$&$2.5 F_{\text{MSY}} \rightarrow F_{\text{MSY}}$&Index SD = 0.4, Age composition SD = 1.5\tabularnewline
$\text{Sel}_{11}$&$0.5$&$0.1$&$0.9$&$2.5 F_{\text{MSY}} \rightarrow F_{\text{MSY}}$&Index SD = 0.4, Age composition SD = 1.5\tabularnewline
$\text{Sel}_{12}$&$0.5$&$0.5$&$0.9$&$2.5 F_{\text{MSY}} \rightarrow F_{\text{MSY}}$&Index SD = 0.4, Age composition SD = 1.5\tabularnewline
$\text{Sel}_{13}$&$0.5$&$0.1$&$0.0$&$F_{\text{MSY}}$&Index SD = 0.4, Age composition SD = 1.5\tabularnewline
$\text{Sel}_{14}$&$0.5$&$0.5$&$0.0$&$F_{\text{MSY}}$&Index SD = 0.4, Age composition SD = 1.5\tabularnewline
$\text{Sel}_{15}$&$0.5$&$0.1$&$0.9$&$F_{\text{MSY}}$&Index SD = 0.4, Age composition SD = 1.5\tabularnewline
$\text{Sel}_{16}$&$0.5$&$0.5$&$0.9$&$F_{\text{MSY}}$&Index SD = 0.4, Age composition SD = 1.5\tabularnewline
\hline
\end{tabular}\end{center}
}
\end{table}

\begin{table}
\caption{Distinguishing characteristics of the OMs with random effects on recruitment and catchability (R+q). When observation uncertainty is low, standard deviations for log-normal distributed indices and logistic normal distributed age composition observations are 0.1 and 0.3, respectively, and when it is high, standard deviations are 0.4 and 1.5, respectively. Fishing mortality either changes from 2.5$F_{\text{MSY}}$ to $F_{\text{MSY}}$ after year 20 (of 40) or is constant at $F_{\text{MSY}}$ over all years. For AR1 process errors, $\sigma_q$ is defined for the marginal distribution of the processes.}\label{q_om_table}
{\begin{center}
\begin{tabular}{rrrrrr}
\hline\hline
\multicolumn{1}{c}{Model}&\multicolumn{1}{c}{$\sigma_R$}&\multicolumn{1}{c}{$\sigma_{q}$}&\multicolumn{1}{c}{$\rho_{q}$}&\multicolumn{1}{c}{Fishing History}&\multicolumn{1}{c}{Observation Uncertainty}\tabularnewline
\hline
$q_{1}$&$0.5$&$0.1$&$0.0$&$2.5 F_{\text{MSY}} \rightarrow F_{\text{MSY}}$&Index SD = 0.1, Age composition SD = 0.3\tabularnewline
$q_{2}$&$0.5$&$0.5$&$0.0$&$2.5 F_{\text{MSY}} \rightarrow F_{\text{MSY}}$&Index SD = 0.1, Age composition SD = 0.3\tabularnewline
$q_{3}$&$0.5$&$0.1$&$0.9$&$2.5 F_{\text{MSY}} \rightarrow F_{\text{MSY}}$&Index SD = 0.1, Age composition SD = 0.3\tabularnewline
$q_{4}$&$0.5$&$0.5$&$0.9$&$2.5 F_{\text{MSY}} \rightarrow F_{\text{MSY}}$&Index SD = 0.1, Age composition SD = 0.3\tabularnewline
$q_{5}$&$0.5$&$0.1$&$0.0$&$F_{\text{MSY}}$&Index SD = 0.1, Age composition SD = 0.3\tabularnewline
$q_{6}$&$0.5$&$0.5$&$0.0$&$F_{\text{MSY}}$&Index SD = 0.1, Age composition SD = 0.3\tabularnewline
$q_{7}$&$0.5$&$0.1$&$0.9$&$F_{\text{MSY}}$&Index SD = 0.1, Age composition SD = 0.3\tabularnewline
$q_{8}$&$0.5$&$0.5$&$0.9$&$F_{\text{MSY}}$&Index SD = 0.1, Age composition SD = 0.3\tabularnewline
$q_{9}$&$0.5$&$0.1$&$0.0$&$2.5 F_{\text{MSY}} \rightarrow F_{\text{MSY}}$&Index SD = 0.4, Age composition SD = 1.5\tabularnewline
$q_{10}$&$0.5$&$0.5$&$0.0$&$2.5 F_{\text{MSY}} \rightarrow F_{\text{MSY}}$&Index SD = 0.4, Age composition SD = 1.5\tabularnewline
$q_{11}$&$0.5$&$0.1$&$0.9$&$2.5 F_{\text{MSY}} \rightarrow F_{\text{MSY}}$&Index SD = 0.4, Age composition SD = 1.5\tabularnewline
$q_{12}$&$0.5$&$0.5$&$0.9$&$2.5 F_{\text{MSY}} \rightarrow F_{\text{MSY}}$&Index SD = 0.4, Age composition SD = 1.5\tabularnewline
$q_{13}$&$0.5$&$0.1$&$0.0$&$F_{\text{MSY}}$&Index SD = 0.4, Age composition SD = 1.5\tabularnewline
$q_{14}$&$0.5$&$0.5$&$0.0$&$F_{\text{MSY}}$&Index SD = 0.4, Age composition SD = 1.5\tabularnewline
$q_{15}$&$0.5$&$0.1$&$0.9$&$F_{\text{MSY}}$&Index SD = 0.4, Age composition SD = 1.5\tabularnewline
$q_{16}$&$0.5$&$0.5$&$0.9$&$F_{\text{MSY}}$&Index SD = 0.4, Age composition SD = 1.5\tabularnewline
\hline
\end{tabular}\end{center}
}
\end{table}

\begin{table}
\caption{Distinguishing characteristics of the EMs and indication (+) of which OM process error sources (R, R+S, R+M, R+Sel, R+q) each EM configuration was fit.}\label{em_table}
{\scriptsize \begin{center}
\begin{tabular}{rrrr}
\hline\hline
\multicolumn{1}{c}{Model}&\multicolumn{1}{c}{Recruitment model}&\multicolumn{1}{c}{Mean $M$}&\multicolumn{1}{c}{Process error assumption}\tabularnewline
\hline
EM$_{1}$&Mean recruitment&0.2&Recruitment ($\sigma_R$ estimated)\tabularnewline
EM$_{2}$&Beverton-Holt&0.2&Recruitment ($\sigma_R$ estimated)\tabularnewline
EM$_{3}$&Mean recruitment&Estimated&Recruitment ($\sigma_R$ estimated)\tabularnewline
EM$_{4}$&Beverton-Holt&Estimated&Recruitment ($\sigma_R$ estimated)\tabularnewline
EM$_{5}$&Mean recruitment&0.2&Recruitment and survival ($\sigma_R$, $\sigma_{2+}$ estimated)\tabularnewline
EM$_{6}$&Beverton-Holt&0.2&Recruitment and survival ($\sigma_R$, $\sigma_{2+}$ estimated)\tabularnewline
EM$_{7}$&Mean recruitment&Estimated&Recruitment and survival ($\sigma_R$, $\sigma_{2+}$ estimated)\tabularnewline
EM$_{8}$&Beverton-Holt&Estimated&Recruitment and survival ($\sigma_R$, $\sigma_{2+}$ estimated)\tabularnewline
EM$_{9}$&Mean recruitment&0.2&Recruitment and uncorrelated natural mortality ($\sigma_R$, $\sigma_{M}$ estimated, $\rho_{M} = 0$)\tabularnewline
EM$_{10}$&Beverton-Holt&0.2&Recruitment and uncorrelated natural mortality ($\sigma_R$, $\sigma_{M}$ estimated, $\rho_{M} = 0$)\tabularnewline
EM$_{11}$&Mean recruitment&Estimated&Recruitment and uncorrelated natural mortality ($\sigma_R$, $\sigma_{M}$ estimated, $\rho_{M} = 0$)\tabularnewline
EM$_{12}$&Beverton-Holt&Estimated&Recruitment and uncorrelated natural mortality ($\sigma_R$, $\sigma_{M}$ estimated, $\rho_{M} = 0$)\tabularnewline
EM$_{13}$&Mean recruitment&0.2&Recruitment and uncorrelated fleet selectivity ($\sigma_R$, $\sigma_{\text{Sel}}$ estimated, $\rho_{\text{Sel}} = 0$)\tabularnewline
EM$_{14}$&Beverton-Holt&0.2&Recruitment and uncorrelated fleet selectivity ($\sigma_R$, $\sigma_{\text{Sel}}$ estimated, $\rho_{\text{Sel}} = 0$)\tabularnewline
EM$_{15}$&Mean recruitment&Estimated&Recruitment and uncorrelated fleet selectivity ($\sigma_R$, $\sigma_{\text{Sel}}$ estimated, $\rho_{\text{Sel}} = 0$)\tabularnewline
EM$_{16}$&Beverton-Holt&Estimated&Recruitment and uncorrelated fleet selectivity ($\sigma_R$, $\sigma_{\text{Sel}}$ estimated, $\rho_{\text{Sel}} = 0$)\tabularnewline
EM$_{17}$&Mean recruitment&0.2&Recruitment and uncorrelated catchability (spring index) ($\sigma_R$, $\sigma_{q}$ estimated, $\rho_{q} = 0$)\tabularnewline
EM$_{18}$&Beverton-Holt&0.2&Recruitment and uncorrelated catchability (spring index) ($\sigma_R$, $\sigma_{q}$ estimated, $\rho_{q} = 0$)\tabularnewline
EM$_{19}$&Mean recruitment&Estimated&Recruitment and uncorrelated catchability (spring index) ($\sigma_R$, $\sigma_{q}$ estimated, $\rho_{q} = 0$)\tabularnewline
EM$_{20}$&Beverton-Holt&Estimated&Recruitment and uncorrelated catchability (spring index) ($\sigma_R$, $\sigma_{q}$ estimated, $\rho_{q} = 0$)\tabularnewline
EM$_{21}$&Mean recruitment&0.2&Recruitment and AR1 natural mortality ($\sigma_R$, $\sigma_{M}$, $\rho_{M}$ estimated)\tabularnewline
EM$_{22}$&Beverton-Holt&0.2&Recruitment and AR1 natural mortality ($\sigma_R$, $\sigma_{M}$, $\rho_{M}$ estimated)\tabularnewline
EM$_{23}$&Mean recruitment&Estimated&Recruitment and AR1 natural mortality ($\sigma_R$, $\sigma_{M}$, $\rho_{M}$ estimated)\tabularnewline
EM$_{24}$&Beverton-Holt&Estimated&Recruitment and AR1 natural mortality ($\sigma_R$, $\sigma_{M}$, $\rho_{M}$ estimated)\tabularnewline
EM$_{25}$&Mean recruitment&0.2&Recruitment and AR1 selectivity ($\sigma_R$, $\sigma_{\text{Sel}}$, $\rho_{\text{Sel}}$ estimated)\tabularnewline
EM$_{26}$&Beverton-Holt&0.2&Recruitment and AR1 selectivity ($\sigma_R$, $\sigma_{\text{Sel}}$, $\rho_{\text{Sel}}$ estimated)\tabularnewline
EM$_{27}$&Mean recruitment&Estimated&Recruitment and AR1 selectivity ($\sigma_R$, $\sigma_{\text{Sel}}$, $\rho_{\text{Sel}}$ estimated)\tabularnewline
EM$_{28}$&Beverton-Holt&Estimated&Recruitment and AR1 selectivity ($\sigma_R$, $\sigma_{\text{Sel}}$, $\rho_{\text{Sel}}$ estimated)\tabularnewline
EM$_{29}$&Mean recruitment&0.2&Recruitment and AR1 catchability (spring index) ($\sigma_R$, $\sigma_{q}$, $\rho_{q}$ estimated)\tabularnewline
EM$_{30}$&Beverton-Holt&0.2&Recruitment and AR1 catchability (spring index) ($\sigma_R$, $\sigma_{q}$, $\rho_{q}$ estimated)\tabularnewline
EM$_{31}$&Mean recruitment&Estimated&Recruitment and AR1 catchability (spring index) ($\sigma_R$, $\sigma_{q}$, $\rho_{q}$ estimated)\tabularnewline
EM$_{32}$&Beverton-Holt&Estimated&Recruitment and AR1 catchability (spring index) ($\sigma_R$, $\sigma_{q}$, $\rho_{q}$ estimated)\tabularnewline
\hline
\end{tabular}\end{center}
}
\end{table}

\begin{table}
\caption{For each OM process error source (columns), percent reduction in deviance for logistic regression models fit to indicators of convergence (maximum absolute gradient < $10^{-6}$) with each OM and EM factor (rows) included individually, combined, and with second and third order interactions.}\label{convergence_gradient_PRD_table}
{\begin{center}
\begin{tabular}{lrrrrr}
\hline\hline
\multicolumn{1}{l}{Factor}&\multicolumn{1}{c}{R}&\multicolumn{1}{c}{R+S}&\multicolumn{1}{c}{R+M}&\multicolumn{1}{c}{R+Sel}&\multicolumn{1}{c}{R+q}\tabularnewline
\hline
EM Process Error&30.40& 0.45&17.57&16.04&24.03\tabularnewline
EM $M$ Assumption& 2.38&24.11& 4.42& 1.02& 2.66\tabularnewline
EM SR Assumption& 1.80& 0.32& 0.96& 3.38& 2.13\tabularnewline
OM Obs. Error& 0.12& 0.77& 0.33& 1.76& 0.28\tabularnewline
OM $F$ History& 3.51& 6.33& 2.36& 5.86& 5.30\tabularnewline
OM $\sigma_R$&\textless  0.01&\textless  0.01&--&--&--\tabularnewline
OM $\sigma_{2+}$ &--&\textless  0.01&--&--&--\tabularnewline
OM $\sigma_M$&--&--& 0.39&--&--\tabularnewline
OM $\rho_M$&--&--& 0.09&--&--\tabularnewline
OM $\sigma_{\text{Sel}}$&--&--&--& 1.08&--\tabularnewline
OM $\rho_{\text{Sel}}$&--&--&--& 0.01&--\tabularnewline
OM $\sigma_q$&--&--&--&--& 0.06\tabularnewline
OM $\rho_q$&--&--&--&--&\textless  0.01\tabularnewline
All factors&43.69&35.72&29.33&34.57&40.69\tabularnewline
+ All Two Way&50.53&42.99&43.91&45.93&48.62\tabularnewline
+ All Three Way&52.30&48.41&46.81&47.71&50.40\tabularnewline
\hline
\end{tabular}\end{center}
}
\end{table}

\begin{table}
\caption{For each OM process error source (columns), percent reduction in deviance for linear regression models fit to errors in estimation measured by Eq. \ref{bias_regression_response} for the terminal year fully-selected fishing mortality with each OM and EM factor (rows) included individually, combined, and with second and third order interactions.}\label{bias_F_PRD_table}
{\begin{center}
\begin{tabular}{lrrr}
\hline\hline
\multicolumn{1}{l}{Factor}&\multicolumn{1}{c}{R}&\multicolumn{1}{c}{R+S}&\multicolumn{1}{c}{R+M}\tabularnewline
\hline
Convergence&\textless  0.01& 0.19& 0.01\tabularnewline
OM $F$ history& 1.92& 2.42& 1.94\tabularnewline
OM Obs. Error& 0.74& 0.12& 1.12\tabularnewline
$OM \sigma_e$& 0.01&\textless  0.01&\textless  0.01\tabularnewline
$OM \sigma_E$&\textless  0.01&\textless  0.01&\textless  0.01\tabularnewline
$OM \rho_{E}$&\textless  0.01&\textless  0.01&\textless  0.01\tabularnewline
OM $\beta_{E}$&\textless  0.01& 0.01&\textless  0.01\tabularnewline
EM Process Error& 0.40& 0.92& 0.48\tabularnewline
$EM \beta_{E}$ assumption& 0.01& 0.03& 0.02\tabularnewline
EM $M$ assumption& 1.58& 1.85& 1.68\tabularnewline
All factors& 4.95& 5.82& 5.35\tabularnewline
+ All Two Way&10.91&11.03&12.23\tabularnewline
+ All Three Way&14.16&13.24&16.19\tabularnewline
\hline
\end{tabular}\end{center}
}
\end{table}

\begin{table}
\caption{For each OM process error source (columns), percent reduction in deviance for linear regression models fit to errors in estimation measured by Eq. \ref{bias_regression_response} for the terminal year recruitment with each OM and EM factor (rows) included individually, combined, and with second and third order interactions.}\label{bias_R_PRD_table}
{\begin{center}
\begin{tabular}{lrrrrr}
\hline\hline
\multicolumn{1}{l}{Factor}&\multicolumn{1}{c}{R}&\multicolumn{1}{c}{R+S}&\multicolumn{1}{c}{R+M}&\multicolumn{1}{c}{R+Sel}&\multicolumn{1}{c}{R+q}\tabularnewline
\hline
EM $M$ Assumption& 1.96& 0.40& 0.69& 3.52& 3.03\tabularnewline
EM SR assumption& 0.06& 0.02& 0.05& 0.02& 0.05\tabularnewline
EM Process Error& 0.39& 4.74& 0.41& 0.12& 1.16\tabularnewline
OM Obs. Error& 1.47& 0.08& 0.64& 0.18&\textless  0.01\tabularnewline
OM $F$ History& 2.54& 2.66& 1.11& 4.18& 5.06\tabularnewline
OM $\sigma_R$& 0.03& 0.01&--&--&--\tabularnewline
OM $\sigma_{2+}$ &--& 1.05&--&--&--\tabularnewline
OM $\sigma_M$&--&--& 0.36&--&--\tabularnewline
OM $\rho_R$&--&--& 0.02&--&--\tabularnewline
OM $\sigma_{Sel}$&--&--&--& 0.23&--\tabularnewline
OM $\rho_{Sel}$&--&--&--& 0.06&--\tabularnewline
OM $\sigma_q$&--&--&--&--& 1.09\tabularnewline
OM $\rho_q$&--&--&--&--& 0.06\tabularnewline
All factors& 6.90& 9.01& 3.43& 8.58&10.90\tabularnewline
+ All Two Way&16.48&24.64& 9.73&15.76&22.75\tabularnewline
+ All Three Way&21.46&35.60&13.56&19.07&31.15\tabularnewline
\hline
\end{tabular}\end{center}
}
\end{table}

\clearpage

\subsection*{Further Results}\label{further-results}
\addcontentsline{toc}{subsection}{Further Results}

\begin{landscape}
\begin{figure}
\begin{center}
\includegraphics[width = 1.4\textwidth]{term_SSB_bias_plots}
\end{center}
\caption{Median relative error of terminal year SSB for EMs fitted to data sets simulated with alternative process error sources: R and R+S (A), R+Sel (B), R+M (C), or R+q (D). Circled values indicate results where the EM process error structure matches that of the OM and vertical lines represent 95\% confidence intervals.}\label{SSB_rel_error}
\end{figure}
\end{landscape}

\begin{landscape}
\begin{figure}
\begin{center}
\includegraphics[width = 1.4\textwidth]{term_F_bias_plots}
\end{center}
\caption{Median relative error of terminal year fully-selected fishing mortality for EMs fitted to data sets simulated with alternative process error sources: R and R+S (A), R+Sel (B), R+M (C), or R+q (D). Circled values indicate results where the EM process error structure matches that of the OM and vertical lines represent 95\% confidence intervals.}\label{F_rel_error}
\end{figure}
\end{landscape}

\begin{landscape}
\begin{figure}
\begin{center}
\includegraphics[width = 1.4\textwidth]{term_R_bias_plots}
\end{center}
\caption{Median relative error of terminal year recruitment for EMs fitted to data sets simulated with alternative process error sources: R and R+S (A), R+Sel (B), R+M (C), or R+q (D). Circled values indicate results where the EM process error structure matches that of the OM and vertical lines represent 95\% confidence intervals.}\label{R_rel_error}
\end{figure}
\end{landscape}

\begin{landscape}
\begin{figure}
\begin{center}
\includegraphics[width = 1.4\textwidth]{M_bias_plots}
\end{center}
\caption{Median relative error of median natural mortality for EMs fitted to data sets simulated with alternative process error sources: R and R+S (A), R+Sel (B), R+M (C), or R+q (D). Circled values indicate results where the EM process error structure matches that of the OM and vertical lines represent 95\% confidence intervals.}\label{M_rel_error}
\end{figure}
\end{landscape}

\begin{landscape}
\begin{figure}
\begin{center}
\includegraphics[width = 1.4\textwidth]{mohns_rho_ssb_plots}
\end{center}
\caption{Median Mohn's $\rho$ for SSB for EMs fitted to data sets simulated with alternative process error sources: R and R+S (A), R+Sel (B), R+M (C), or R+q (D). Circled values indicate results where the EM process error structure matches that of the OM and vertical lines represent 95\% confidence intervals.}\label{mohns_rho_ssb}
\end{figure}
\end{landscape}

\begin{landscape}
\begin{figure}
\begin{center}
\includegraphics[width = 1.4\textwidth]{mohns_rho_F_plots}
\end{center}
\caption{Median Mohn's $\rho$ of fishing mortality averaged over all age classes for EMs fitted to data sets simulated with alternative process error sources: R and R+S (A), R+Sel (B), R+M (C), or R+q (D). Circled values indicate results where the EM process error structure matches that of the OM and vertical lines represent 95\% confidence intervals.}\label{mohns_rho_F}
\end{figure}
\end{landscape}

\begin{landscape}
\begin{figure}
\begin{center}
\includegraphics[width = 1.4\textwidth]{mohns_rho_R_plots}
\end{center}
\caption{Median Mohn's $\rho$ of recruitment for EMs fitted to data sets simulated with alternative process error sources: R and R+S (A), R+Sel (B), R+M (C), or R+q (D). Circled values indicate results where the EM process error structure matches that of the OM and vertical lines represent 95\% confidence intervals.}\label{mohns_rho_R}
\end{figure}
\end{landscape}

\begin{landscape}
\begin{figure}
\begin{center}
\includegraphics[width = 1.4\textwidth]{type_4_convergence_plots}
\end{center}
\caption{Probability of EMs providing Hessian-based standard errors with alternative process error (colored points and lines), and median natural mortality (estimated or known) and Beverton-Holt SRR (estimated or not; along x-axis) assumptions when fitted to OMs that have R and R+S (A), R+Sel (B), R+M (C), or R+q (D) process error sources. Circled values indicate results where the EM process error structure matches that of the OM and vertical lines represent 95\% confidence intervals.}\label{hessian_SE_convergence}
\end{figure}
\end{landscape}

\begin{landscape}
\begin{figure}
\begin{center}
\includegraphics[width = 1.4\textwidth]{type_3_convergence_plots}
\end{center}
\caption{Probability of EMs providing maximum absolute values of gradients less than $10^{-6}$ with alternative process error (colored points and lines), and median natural mortality (estimated or known) and Beverton-Holt SRR (estimated or not; along x-axis) assumptions when fitted to OMs that have R and R+S (A), R+Sel (B), R+M (C), or R+q (D) process error sources. Circled values indicate results where the EM process error structure matches that of the OM and vertical lines represent 95\% confidence intervals.}\label{gradient_convergence}
\end{figure}
\end{landscape}

\begin{landscape}
\begin{figure}
\begin{center}
\includegraphics[height = 0.8\textheight]{sr_aic_plots_rev}
\end{center}
\caption{Probability of lowest AIC from logistic regression on the log-standard deviation of the true log(SSB) in each simulation for EM with Beverton-Holt SRRs, rather than the otherwise equivalent EM without the SRR. Results are conditional on median M is known in the EM and alternative assumptions EMs having the correct process error structure: R and R+S (A), R+Sel (B), R+M (C), or R+q (D), and median M is assumed known in the EM. Solid and dashed lines are for OMs with and without temporal contrast in fishing pressure, respectively, and polygons represent 95\% confidence intervals. Range of results indicates the range of log-standard deviation of log(SSB) for simulations of the particular OM.}\label{sr_aic}
\end{figure}
\end{landscape}

\begin{landscape}
\begin{figure}
\begin{center}
\includegraphics[width = 1.4\textwidth]{sr_aic_plots}
\end{center}
\caption{Estimated probability of lowest AIC from logistic regression on the log-standard deviation of the true log(SSB) in each simulation for EM with Beverton-Holt SRRs, rather than the otherwise equivalent EM without the SRR. Results are conditional on alternative assumptions for median natural mortality (estimated or known) and on EMs having the correct process error structure: R and R+S (A), R+Sel (B), R+M (C), or R+q (D). Rug along x-axis denotes $SD(\log(SSB))$ values for each simulation and polygons represent 95\% confidence intervals.}\label{sr_aic_supp}
\end{figure}
\end{landscape}

\begin{table}
\caption{For each OM process error source (columns), percent reduction in deviance for linear regression models fit to transformed Mohn's $\rho$ values for each simulation (Eq. \ref{bias_regression_response}) for fishing mortality averaged over all age classes with each OM and EM factor (rows) included individually, combined, and with second and third order interactions.}\label{mohns_rho_F_PRD_table}
{\begin{center}
\begin{tabular}{lrrrrr}
\hline\hline
\multicolumn{1}{l}{Factor}&\multicolumn{1}{c}{R}&\multicolumn{1}{c}{R+S}&\multicolumn{1}{c}{R+M}&\multicolumn{1}{c}{R+Sel}&\multicolumn{1}{c}{R+q}\tabularnewline
\hline
EM $M$ Assumption&0.06&0.09&0.01&0.12&0.01\tabularnewline
EM SR assumption&0.01&\textless  0.01&0.01&0.02&0.01\tabularnewline
EM Process Error&0.03&0.07&0.02&0.06&0.03\tabularnewline
OM Obs. Error&0.16&0.10&0.05&0.02&0.07\tabularnewline
OM $F$ History&0.07&0.02&0.03&0.24&0.03\tabularnewline
OM $\sigma_R$&\textless  0.01&0.01&--&--&--\tabularnewline
OM $\sigma_{2+}$ &--&0.09&--&--&--\tabularnewline
OM $\sigma_M$&--&--&\textless  0.01&--&--\tabularnewline
OM $\rho_R$&--&--&\textless  0.01&--&--\tabularnewline
OM $\sigma_{Sel}$&--&--&--&0.01&--\tabularnewline
OM $\rho_{Sel}$&--&--&--&\textless  0.01&--\tabularnewline
OM $\sigma_q$&--&--&--&--&\textless  0.01\tabularnewline
OM $\rho_q$&--&--&--&--&0.01\tabularnewline
All factors&0.32&0.38&0.12&0.48&0.15\tabularnewline
+ All Two Way&0.65&0.67&0.30&0.95&0.43\tabularnewline
+ All Three Way&1.18&1.11&0.63&1.34&0.90\tabularnewline
\hline
\end{tabular}\end{center}
}
\end{table}

\begin{table}
\caption{For each OM process error source (columns), percent reduction in deviance for linear regression models fit to transformed Mohn's $\rho$ values for each simulation (Eq. \ref{bias_regression_response}) for recruitment with each OM and EM factor (rows) included individually, combined, and with second and third order interactions.}\label{mohns_rho_R_PRD_table}
{\begin{center}
\begin{tabular}{lrrrrr}
\hline\hline
\multicolumn{1}{l}{Factor}&\multicolumn{1}{c}{R}&\multicolumn{1}{c}{R+S}&\multicolumn{1}{c}{R+M}&\multicolumn{1}{c}{R+Sel}&\multicolumn{1}{c}{R+q}\tabularnewline
\hline
EM $M$ Assumption&0.86&0.56&0.16&1.00&1.27\tabularnewline
EM SR assumption&\textless  0.01&0.02&0.01&0.01&0.01\tabularnewline
EM Process Error&0.01&0.59&0.18&0.07&0.04\tabularnewline
OM Obs. Error&0.34&0.01&0.08&0.24&0.27\tabularnewline
OM $F$ History&0.91&0.22&0.06&1.20&1.67\tabularnewline
OM $\sigma_R$&\textless  0.01&0.14&--&--&--\tabularnewline
OM $\sigma_{2+}$ &--&0.11&--&--&--\tabularnewline
OM $\sigma_M$&--&--&0.01&--&--\tabularnewline
OM $\rho_R$&--&--&\textless  0.01&--&--\tabularnewline
OM $\sigma_{Sel}$&--&--&--&0.01&--\tabularnewline
OM $\rho_{Sel}$&--&--&--&0.01&--\tabularnewline
OM $\sigma_q$&--&--&--&--&0.01\tabularnewline
OM $\rho_q$&--&--&--&--&0.01\tabularnewline
All factors&2.28&1.74&0.51&2.66&3.51\tabularnewline
+ All Two Way&4.20&2.74&1.08&5.08&6.51\tabularnewline
+ All Three Way&4.83&3.79&1.79&6.03&7.82\tabularnewline
\hline
\end{tabular}\end{center}
}
\end{table}

\end{document}
